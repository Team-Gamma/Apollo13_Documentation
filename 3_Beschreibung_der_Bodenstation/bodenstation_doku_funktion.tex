\subsection{Funnktion}
\subsubsection{Nutzerfreundlichkeit}



\subsubsection{Erweiterbarkeit}


\subsubsection{Features}
Da die Software der Bodenstation auf dem Framework ´´Netbeans Plantform`` basiert, lassen sich einzelne graphische Module kombinieren, welche sich per drag and drop verschieben lassen. Die Größe und Position dieser Module und des gesamten Frames lassen sich beliebig verändern. Des weiteren bietet die Software die Möglichkeit Daten von verschiedenen Sateliten zu empfangen. Empfangene Daten lassen sich mittels der graphischen Oberfläche in Graphen anzeigen, welche sich verschiedens kombinieren lassen. Außerdem lassen sich die empfangenen Daten zwischenspeichern und anschließend in verschiedene Dateiformate Exportieren oder live in einer Datei loggen. Diese exportierten oder geloggten Daten lassen sich anschließend wieder einlesen und Anzeigen. CSV, Txt, JSON, KML und png sind Dateiformate, welche exportierbar sind. Davon lassen sich exportierte CSV und JSON Datein wieder einlesen und Visualisieren. Txt Dateien werden Formatiert und somit gut leserlich für den Leser exportiert, während exportierte KML-Dateien per Google Earth geöffnet und graphisch Visualisiert werden können. Ein weiteres Feature der Bodenstationsoftware ist die Datenvisuaisierung per ``Nase World Wind´´ als Modul in der graphischen Oberfläche. Diese Visualisierung zeigt den Flug des Sateliten auf einem virtuellen Globus, mittels Satelliten- und Luftbilder, und zeigt auf jeder gemessenen GPS Koordinate die gemessenen Werte der Sensoren des Sateliten. Unter anderem bietet dieses Modul der Software die Möglichkeit an der Virtuellen Erdkugel heran zu zoomen und einzelne Elemente dreidimensional darzustellen. Darstellungen mittels dieses virtuellen Erdballs sind sowohl in Echtzeit mittels eines Streams vom Satelliten als auch als import aus einer Datei möglich.
