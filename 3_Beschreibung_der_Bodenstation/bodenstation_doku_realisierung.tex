\subsection{Realisierung}
Während der Realisierung der Bodenstation kam es zu einigen Komplikationen. Zum einen wurde die Softwarearchitektur während der Implementationsphase geändert, so dass verschiedene Komponenten wie zum Beispiel Export- und Importkomponenten mehrfach realisiert wurden. Diese Architekturveränderung wurde vorgenommen und gewisse Komplikationen zu beseitigen, welche die Modulare Strukturierung von Netbeans Plattform mit sich bringt. Wir starteten mit einer Architektur, welche jede wichtige Komponente als Modul benennt. Diese Architektur brachte zum einen das Problem, dass Abhängigkeiten zwischen Modulen nur in eine Richtung stattfinden kann. Die neue Architektur unterscheidet lediglich zwischen den Modulen API, GUI und Core. Des weiteren wurden verwendete Datentypen innerhalb der Software während der Implementationsphase geändert, so dass zusätzlich alle Komponenten, welche mit der Datenverarbeitung Zutun haben geändert werden mussten. Darunter vielen die gesamten Komponenten des Imports, Exports und der Live-Datenverarbeitung. Die Veränderung der benutzten Datentypen wurde von Long, String und Double zu einzig Double geändert, da alle Daten, welche von kompatiblem Satelliten in Double dargestellt werden können. Diese Änderung im Programmcode hat zwar einiges an Arbeit gekostet, doch der Umfang des Programmcodes wurde deutlich verringert und die Performance gesteigert.