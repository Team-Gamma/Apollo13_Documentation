\subsubsection{Tests}
\paragraph{Automatisierte Tests}
Insgesamt wurde lediglich die Input-Pipeline mithilfe von JUnit automatisiert getestet, da diese eine komplizierten Programmcode enthält.
Dabei werden bei den jeweiligen Export-Komponenten jeweils Testdaten in das jeweilige Format exportiert. Währenddessen kann überprüft werden, ob sich die Komponente bei der Übergabe verschiedener Parameter wie geplant verhält. Außerdem kann geprüft werden, ob die erzeugten oder veränderten Dateien wie geplant aussehen. Darüber hinaus wurde für jeden Import-Komponenten ein automatisierter Test geschrieben, welche den Komponenten auf das Verhalten bei verschiedenen Parametern überprüft. Des Weiteren werden Daten erzeugt, welche zunächst mit dem jeweiligen Komponenten exportiert werden und anschließend mit dem passenden Komponenten importiert werden. Dabei wird geprüft, ob sich die importierten Daten von den ursprünglichen Daten unterscheiden.

\paragraph{Manuelle Tests}
Softwarekomponenten, welche nicht durch automatisierten Softwaretests getestet wurden, wurden manuell getestet. Beispielsweise wurde der Socket-Server, welcher empfangene Daten an Clients versendet, manuell getestet. Um diesen zu testen, wurde eine Verbindung zu mehreren Clients aufgebaut und Daten an diese versendet. Außerdem wurden einige mögliche Szenarien, in denen sich die Bodenstation befinden kann, ausprobiert. Gefundene Fehler wurden anschließend behoben. Alle graphischen Komponenten wurden manuell getestet, dabei wurden mögliche Nutzerszenarien simuliert, um mögliche Fehler zu finden.