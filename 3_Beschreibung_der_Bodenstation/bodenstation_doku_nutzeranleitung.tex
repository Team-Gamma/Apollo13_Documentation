\subsubsection{Nutzeranleitung}
\paragraph{Datenempfang}
Um Daten in Echtzeit zu empfangen muss eine Verbindung zu einem Satelliten aufgebaut werden. Um diese Verbindung aufzubauen.........
\paragraph{Datenimport}
Um Daten aus einer Datei zu importieren wird Im Menüpunkt File der Untermenüpunkt Import Ausgewählt. Anschließend wird eine Datei ausgewählt, welche importiert werden soll. Mit der Bestätigung werden die Daten dieser Datei eingelesen.
\paragraph{Graphenkonfiguration}
\paragraph{Datenexport}
Für das Exportieren der Daten gilt, dass alle aktuell geladenen Daten exportiert werden. Darunter fallen entweder zwischengespeicherte Daten einer Liveübertragung oder Daten, welche aus einer Datei importiert werden.
Zum Exportieren der Daten wird im Menüpunkt File der Untermenüpunkt Export ausgewählt. Unter diesem Menüpunkt ist das Datenformat wählbar, in welches die gesammelten Daten gespeichert werden. Anschließend ist ein Pfad und eine Name wählbar unter dem die Datei gespeichert wird. Mit der Bestätigung werden die Daten exportiert.
\paragraph{Datenweiterleitung}
\paragraph{Oberflächenpersonalisierung}
Der Oberfläche können einzelne Komponenten hinzugefügt und entfernt werden. Diese Komponenten können unterschiedlich angeordnet werden. Um Komponenten wie z.B einen Graph hinzuzufügen......... Um einen der bestehenden Komponenten zu entfernen............... Per "drag and drop" können diese Komponenten neu angeordnet werden. In verschiedenen Bereichen können Komponenten angeheftet oder verschoben werden. Außerdem können diese übereinander verlagert werden um diese in verschiedenen Tabs zu verwalten.