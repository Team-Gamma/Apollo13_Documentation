\section{Projektplanung}
\subsection{Zeitplan der CanSat-Vorbereitung}
Insgesamt wurde die Zeitplanung an verschiedenen Meilensteinen orientiert, welche sich auch in den spezifischen Zeitplanungen der Teilgruppen wiederfinden lassen. So war zum Beispiel geplant, einen guten Zwischenstand mit einem Prototypen bereits im Mai zu erreichen, wobei dieser Zeitplan von Anfang an kritisch gesehen wurde und letzten Endes auch nicht erreicht wurde. Trotzdem blieb nach diesem Termin noch genügend Zeit, das Projekt fertigzustellen, was leider letzten Endes die geplante Testphase etwas verkürzt. \\
Das gesamte Management der Arbeitspakete und des Zeitaufwandes wurde mit der Projektmanagementsoftware \href {www.redmine.org} {Redmine} erledigt. Da diese auf unserem Server unter \href{http://redmine.gamma-team.de}{redmine.gamma-team.de} erreichbar ist, kann jedes Teammitglied zu jedem Zeitpunkt den Fortschritt der Arbeit verfolgen. Die Planung der beiden Halbgruppen ist größtenteils voneinander getrennt. Es gibt jedoch gemeinsame Meilensteine, welche von beiden Gruppen eingehalten werden sollen. Bevor die Arbeit der Halbgruppen begonnen hat, gab es eine allgemeine Projektfindungsphase. In dieser Phase wurde ein grober Zeitplan festgelegt und es wurden alle relevanten Systeme (Webserver, Projektmanagementsoftware, GitLab etc.) aufgesetzt und eingerichtet um später einen reibungslosen Ablauf der Arbeitsphase zu garantieren. Die Idee und die Spezialisierung der Idee für das gesamte Projekt entstand ebenfalls in dieser Zeit. Anschließend wurde eine separate Zeitplanung in den beiden Halbgruppen erstellt, welche im Nachfolgenden erläutert wird.

\subsubsection{Zeitplan der Hardware-Gruppe}
Innerhalb der Hardwaregruppe wurde versucht, die meisten Aufgaben zu parallelisieren. Jedes Teammitglied hat sein eigenes spezielles Aufgabengebiet. Zwischen den Teammitgliedern herrscht trotzdem ein stetiger Austausch. Grund für die Parallelisierung war, dass in unseren Augen die meisten Aufgaben  nur die Aufmerksamkeit einer Person benötigen. Es ist nur selten erforderlich, dass mehrere Teammitglieder an demselben Arbeitspaket arbeiten müssen. Der gesamte Arbeitsprozess wurde in diverse Abschnitte gegliedert. Diese Abschnitte lassen sich auch im GANTT-Diagramm im Anhang dieses Dokumentes wiederfinden. Die einzelnen Abschnitte sind in diverse Arbeitspakete unterteilt, Personen zugewiesen und mit einem Zeitraum versehen. Bei den Abschnitten handelt es sich um folgende:
\begin{itemize}
\item Planung: Erstellung von Arbeitspaketen sowie eine Verteilung dieser und eine Erstellung diverser Diagramme
\item Fallschirm: Gestaltung und Bau des Bergungssystems
\item Sensorik: Heraussuchen, Bestellen und Testen passender Sensoren für unser Projekt
\item Beagleboard: Festlegung der Programmiersprache, IDE und der Recherche zu den elektrotechnischen Eigenschaften des Boards
\item Dose: Design und Bau der Hülle und der Deckel der Dose
\item Dosenmanagement: Design und Bau des Inneren der Dose sowie die Integration der Sensoren in das Gesamtsystem
\end{itemize}

Das GANTT-Diagramm der Hardware-Gruppe kann auch unter \ref{gantt_hardware} gefunden werden.

\subsubsection{Zeitplan der Software-Gruppe}
In der Softwaregruppe haben wir uns, wie in der Hardwaregruppe, dafür entschieden, die Aufgaben untereinander zu verteilen. Dabei haben wir zuerst die Bodenstation und die Android-Applikation komplett voneinander getrennt. Die Bodenstation war von Beginn an fester Bestandteil unseres Projektes. Die Android-App kam erst später hinzu und musste daher separiert behandelt werden. Noch vor dem eigentlichen Start des Projektes wurde diskutiert, wie die Software aufgebaut sein soll und welche Technologien für die Bodenstation verwendet werden kann. Nachdem die ersten Entscheidungen getroffen waren, haben wir angefangen, das Projekt in verschiedene Meilensteine zu unterteilen. Daraus ist folgende Gliederung entstanden:
\begin{itemize}
    \item Erstellung einer detaillierten Ticketübersicht
    \item Basisversion, mit allen Tickets, welche zur Bereitstellung der ersten Features nötig sind
    \item Export, mit allen Tickets, welche zum Exportieren von Daten nötig sind
    \item Kartenvisualisierung, mit allen Tickets, welche für das Darstellen des Satellitenfluges nötig sind
    \item Wissenschaftliche Analyse, mit allen Tickets, welche zur wissenschaftlichen Analyse der gesammelten Daten nötig sind
    \item Finale Version, mit allen Tickets, welche zur Zusammenführung der Bodenstation nötig sind
    \item RC 1 v1.1, mit allen Tickets, welche für den letzten Schliff der Bodenstation nötig sind
\end{itemize}
Innerhalb dieser Meilensteine haben wir das Projekt daraufhin in einzelne Tickets unterteilt. Jedes dieser Tickets spiegelt eine einzelne Aufgabe zur Fertigstellung der Bodenstation wieder. Diese Aufteilung hat es uns ermöglicht, die Aufgaben innerhalb der Bodenstation relativ flexibel zu verteilen. Dies hat dazu beigetragen, dass selten jemand auf eine andere Aufgabe warten musste. Die Tickets, welche sich innerhalb der Meilensteine befinden, haben sich während der Durchführung des Projektes immer wieder verändert. Dies war abhängig von den Ansprüchen, welche sich in dem jeweiligen Moment ergeben haben. Das im Anhang unter \ref{gantt_software} vorhandene GANTT-Diagramm gibt also nicht nur unsere Planung zum Anfang des Projektes wieder, sondern auch die kontinuierliche Präzisionsplanung während der Durchführung des Projektes.

Die Idee der App entstand deutlich später als die des restlichen Projektes. Daher wurde bei der Planung in Wochenschritten gedacht. Durch diese Methode konnten alle zwei Wochen kontrolliert werden, ob eine Komponente fertiggestellt ist. Falls mehr Zeit benötigt wurde, so war es möglich, am Wochenende an der Android-App zu arbeiten. Zusammengefasst existieren insgesamt sieben Arbeitspakete:
\begin{itemize}
	\item Der Debugger, der die Livedaten simulieren soll
	\item Liniengraph GUI
	\item Liniengraph Logic
	\item Balkendiagramm GUI
	\item Balkendiagramm Logic
	\item Optionen
	\item Menü
\end{itemize}

Das GANTT für die Android-Applikation lässt sich unter \ref{gantt_android} finden.

\subsection{Einschätzung der Mittel}
\subsubsection{Budget}
% \begin{tabular}{p{1,5cm}p{1,5cm}p{3,5cm}p{6,5cm}rrrl}
\label{subsubsec:Budget}

Um das CanSat Projekt zu finanzieren, konnten wir aktuell noch keine Sponsoren finden. Jedoch konnten wir uns mit unserem Schulverein verständigen, welcher uns finanziell unterstützen wird. Da wir nicht auf das T-Minus-Kit zurückgreifen, sondern stattdessen ein anderes Mikrokontrollerboard verwenden, können wir ungefähr 150\euro  sparen. Der 200\euro  Watterot Gutschein, welcher vom Wettbewerb gestellt wird, ist in unserer Rechnung noch nicht inbegriffen. Dies liegt daran, dass noch nichts bei Watterot bestellt wurde, bzw. die Bestellung lange vor der Annahme am Wettbewerb getätigt wurde.
Im Nachfolgenden sind alle Ausgaben und Einnahmen aufgelistet.
\begin{table}[H]
  \centering
    \begin{tabular}{p{1,7cm}p{1,5cm}p{3,5cm}p{6,5cm}rrrl}
    \toprule
    \textbf{Ausgabe} & \textbf{Datum} & \textbf{Empfänger} & \textbf{Grund} \\
    \midrule
    -12,16 \euro  & 08.01.2015 & Watterott & BMP180 Breakout \\
    -28,99 \euro  & 09.01.2015 & eBay - rcskymodel & Ultimate GPS \\
    -14,32 \euro  & 10.01.2015 & Spark Fun Electronics & UV-Sensor \\
    -51,99 \euro  & 10.01.2015 & Amazon & BeagleBone Black \\
    -17,30 \euro  & 01.12.2014 & eBay - hdt-preiswert & 
GFK-Set 1kg Polyesterharz + 20g Härter + $2m^2$ Glasfasermatte \\
    -3,54 \euro  & 23.03.2015 & toom baumarkt & 6 x Schleifpapier \\
    -3,79 \euro  & 23.03.2015 & toom baumarkt & Filzrolle \\
    -4,49 \euro  & 23.03.2015 & toom baumarkt & Plüschwalzen \\
    -2,19 \euro  & 23.03.2015 & toom baumarkt & Mundschutz \\
    -1,99 \euro  & 23.03.2015 & toom baumarkt & Farbwanne \\
    -4,99 \euro  & 23.03.2015 & toom baumarkt & Einmalhandschuhe \\
    -133,04 \euro  & 02.06.2015 & Alexander Brennecke& Rückzahlung bisheriger Ausgaben\\
	-139,80 \euro  & 23.03.2015 & Amazon & 2x BeagleBone Black\\
    \bottomrule
    - 145,75 \euro & & & \\
    \bottomrule
    \end{tabular}%
    \caption{Ausgaben}
  \label{tab:budgetausgaben}%
\end{table}%

\begin{table}[htbp]
  \centering
    \begin{tabular}{p{1,7cm}p{1,5cm}p{3,5cm}p{6,5cm}rrrl}
    \toprule
    \multicolumn{1}{c}{\textbf{Einnahmen}} & \textbf{Datum} & \textbf{Absender} & \textbf{Grund} \\
    \midrule
              17,30 \euro  & 01.12.2014 & Alexander Brennecke & GFK-Kauf \\
           107,46 \euro  & 10.01.2015 & Alexander Brennecke & Sensorenkauf \\
              20,99 \euro  & 23.03.2015 & Alexander Brennecke & toom Einkauf \\
133,04 \euro  & 02.06.2015 & Schulferein der Europaschule SII Utbremen & Sponsoring \\
    \bottomrule
    145,75 \euro & & & \\
    \bottomrule
    \end{tabular}%
	\caption{Einnahmen}
  \label{tab:budgeteinnahmen}%
\end{table}%

\subsubsection{Externe Unterstützung}
Externe Unterstützung erhielten wir von vielen Lehrern unserer Schule, welche uns Fragen zur Elektrotechnik und Softwareprogrammierung beantworten konnten. Zusätzlich haben wir finanzielle Unterstützung durch den Schulverein unserer Schule erhalten (siehe~\ref{subsubsec:Budget}).
Unterstützung außerhalb unserer Schule erhielten wir durch folgende Personen/Organisationen:

\begin{itemize}
	\item Das \href{https://www.hackerspace-bremen.de/}{Hackerspace Bremen e.V.}, welches uns ihren 3D-Drucker zur Verfügung gestellt hat. Zusätzlich konnten wir dort unsere Platine ätzen.
	\item \href{http://de.wikipedia.org/wiki/Martin_Schneider_(Nachrichtentechniker)} {Prof. Martin Schneider} vom Hochfrequenzlabor der Universität Bremen, welcher uns geholfen hat, unsere Antenne an die Frequenz und die Wellenimpedanz anzupassen.
	\item Das Umweltlabor der \href{http://www.atlas-elektronik.com/atlas-elektronik/}{Atlas Elektronik GmbH} hat uns geholfen, den CanSat, hinsichtlich seiner Stabilität, zu testen und die Sensoren korrekt zu kalibrieren.
\end{itemize}