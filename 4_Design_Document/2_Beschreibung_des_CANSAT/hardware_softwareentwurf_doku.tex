\subsection{Softwaredesign}
\subsubsection{Python als Programmiersprache}
Als Programmiersprache für das BeagleBone Black haben wir uns für Python entschieden. Es wäre ebenfalls möglich gewesen, den Mikrocontroller mit den Sprachen JavaScript, Java, C, C++, C\# und vielen weiteren Sprachen zu programmieren. Da es sich bei dem BeagleBone um ein Linux-basiertes Embedded System handelt, unterstützt es praktisch alle Programmiersprachen, sofern entsprechende Bibliotheken existieren. Allerdings haben wir uns aufgrund der Tatsache, dass Python im Gegensatz zu Java nicht objektorientiert geschrieben werden muss, für Python entschieden. Wir möchten auf der Hardwareseite möglichst auf objektorientierte Programmierung verzichten. Ein weiteres wichtiges Argument ist die gute Python-Bibliothek, welche von einer großen Community permanent gewartet und aktualisiert wird.
\subsubsection{Datenverarbeitung auf dem BeagleBone}
Die Datenverarbeitung auf dem BeagleBone verläuft relativ simpel. Zunächst werden alle von den Sensoren aufgezeichneten Daten, bei Sensoren mit I2C-Anbindung, mithilfe von Libraries, und bei den anderen mithilfe von Umrechnungsalgorithmen, gesammelt. Anschließend werden alle gesammelten Daten in einen JSON-String geparsed, welcher mithilfe von unserem Transceiver an die Bodenstation übermittelt wird. Diese übernimmt hieraufhin die weitere Verarbeitung und Darstellung der Messdaten. Zusätzlich werden die Daten auf dem internen Speicher des BeagleBones gespeichert. \\
Eine simple Skizze der Architektur lässt sich auch unter \ref{blockdiagramm} finden.