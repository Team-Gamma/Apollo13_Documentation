\subsubsection{Kurzbeschreibung}
Die Funktion der Android-App ist es, die Datenpakete, die von der Bodenstation über einen Hotspot gesendet werden, zu empfangen und live, in einem passenden Graphen, anzuzeigen. Dabei wird Wert darauf gelegt, dass es möglich ist, alle Werte die gesendet werden, einzeln oder in Gruppen darzustellen. Dies ermöglicht, dass alle Daten verglichen werden können. Zusätzlich wird in einem Balkendiagramm die Differenz zwischen dem am höchsten und dem niedrigsten Punkt gemessenen Wert dargestellt. Diese Differenzen werden in grün für positive Veränderungen und in rot für negative Veränderung dargestellt werden. Nebenher sind ebenfalls Optionsmöglichkeiten vorhanden, um die Graphen nach den Wünschen des Nutzern zu gestalten.

\subsubsection{Funktionen}
Die App bietet insgesamt folgende Funktionen:
\begin{itemize}
	\item Anzeigen der Werte in einem Livegraphen
	\item Verwaltung der angezeigten Werte im Graphen während der Laufzeit
	\item Anzeigen von Differenzen von Werten im Balkendiagramm
	\item Manuelle Start/Stop Funktion für das Balkendiagramm
	\item Einstellung zur Geschwindigkeit des Graphen
	\item Einstellung zur Regelung der Anzahl der Werte, welche gleichzeitig angezeigt werden
	\item Die Möglichkeit, alle Funktionen mit einem Debug-Stream zu testen
\end{itemize}
