\section{Beschreibung des CanSat}


% Blockdiagramm (irgendjemand)
\subsection{Missionsüberblick}
Wir haben uns für den Satelliten überlegt, dass dieser so individuell wie möglich sein soll. Daher greifen wir nicht auf das vom Wettbewerb bereitgestellte T-Minus-CanSat-Kit zurück. Stattdessen haben wir uns im Detail überlegt, welche Sensoren unseren Erwartungen entsprechen und wie wir diese bestmöglich innerhalb der Dose platzieren können. Zusätzlich möchten wir nicht auf eine standardisierte Getränkedose als Hülle zurückgreifen, sondern auch hier unser eigenes Design erschaffen.

% 3D Skizze, Erklärung der einzelnen Bestandteile (Alexander B.)
\subsection{Mechanik- und Strukturdesign}

Wir haben den CanSat in drei Komponenten aufgeteilt: Die Hülle, die Innenwand und die Sensorikplatine. Diese drei Komponenten bilden den Hauptbestandteil des CanSats und haben maßgeblich zu dem mechanischen und strukturellem Design beigetragen. Im nachfolgenden wird kurz auf jeden dieser Komponenten eingegangen und die exakte Funktion im Zusammenhang erklärt.

\subsubsection{Fachliche Grundlagen}
Um die 3D-gedruckte Wand zu erzeugen, wurde die 3D-Moddelierungssoftware \href{http://www.sketchup.com/de} {Sketchup} von Google verwendet. Sketchup bietet die Möglichkeit, vergleichsweise einfach 3D-Modelle zu zeichnen. Um dies zu tun, muss klar sein, welche Objekte gezeichnet werden sollen. Diese Objekte müssen vermessen und innerhalb von Sketchup gezeichnet werden. Dies erfordert die Kenntnis über gewisse mathematische Methoden zur Berechnung von Kreisen, Flächen und Körpern. Die meisten 3D-Drucker benötigen Dateien des Typs .stl, welche in Sketchup mit einem Plugin erzeugt werden können.
Zum Anfertigen von GFK-Komponenten wird ein Körper benötigt, auf welchen das GFK laminiert werden kann. In unserem Fall ist dieser Körper zylindrisch, mit einem Durchmesser von 31,5 mm, und aus Aluminium gefräst. \\
Um die Platine zu erstellen, wurde die Design-Software \href{http://www.cadsoft.de/eagle-pcb-design-software/} {Eagle PCB} verwendet. Eagle bietet die Möglichkeit, sowohl Schaltpläne als auch das entsprechende Layout zu erstellen. Im Anschluss wurde die Platine, mit Hilfe und Mitteln des \href{https://www.hackerspace-bremen.de/}{Hackerspace Bremen e.V.}, geätzt.

\subsubsection{Hülle}
Wir haben uns dazu entschieden, die äußere Hülle aus GFK (Glasfaser verstärkter Kunststoff) anzufertigen. Dieser hat die Eigenschaft, dass er bei einem sehr geringen Gewicht, und bei einer geringen Wandstärke, trotzdem eine gewisse Stabilität aufweist. Aus dem GFK haben wir eine Röhre mit einem Innendurchmesser von 31,5 mm und einem Außendurchmesser von 33,5 mm laminiert. Diese Röhre wurde auf eine Länge von 111 mm gekürzt und gefeilt. Um die Röhre oben und unten zu verschließen, haben wir uns bei \href{http://www.thyssenkrupp-system-engineering.com/de/home.html}{Thyssen Krupp System Engineering} zwei Aluminiumdeckel fräsen lassen. Diese haben uns ebenfalls durch ihr geringes Gewicht und ihre hohe Stabilität überzeugt. Die Deckel haben, genauso wie die Hülle, einen Außendurchmesser von 33,5 mm. Sie sind beide 2 mm dick und haben zusätzlich eine 2mm dicke Erhöhung, welche in die Röhre eingelassen wird.

\subsubsection{Innenwand}
Um die Elektronik innerhalb der Hülle zu platzieren und zu befestigen, haben wir uns dazu entschieden, eine Wand anzufertigen. Diese Wand teilt die Hülle mittig und bietet so auf beiden Seiten Platz, um unser Mikrokontrollerboard und unsere Sensorikplatine zu befestigen. Beide Bauteile werden mittels vier Gewindestangen an der Wand befestigt. Durch die Technik des 3D-Druckens ist es möglich, der Wand ein sehr geringes Gewicht bei einer verhältnismäßig hohen Stabilität zu verleihen. Zusätzlich gibt es uns die Möglichkeit, die Wand millimetergenau zu gestalten. \\
Am unteren Ende der Wand befindet sich eine Aushöhlung, sowie ein Fuß. Diese ist zum einen dafür da, um den Sharp-Feinstaubsensor zu befestigen. Zum anderen gibt der Fuß der Wand, und somit dem gesamten Satelliten, eine gewisse Stabilität. Der Fuß besitzt auf der einen Seite der Wand Bohrungen. Diese Bohrungen werden verwendet, um die Aluminiumdeckel an der Wand zu befestigen. An der oberen Seite der Wand befinden sich ebenfalls solche Bohrungen, um den oberen Deckel der Hülle zu befestigen. Da der Feinstaubsensor einen Luftzug benötigt, gibt es eine Bohrung, welche vertikal durch die Wand führt. Um das Mikrokontrollerboard mit der Sensorikplatine zu verbinden, existiert ein Fenster in der Mitte der Wand. Um die Sensorikplatine und das Mikrokontrollerboard an der Wand zu befestigen, existieren in der Wand weitere vier Bohrungen.

\subsubsection{Sensorikplatine}
Die Sensorikplatine ist eine von uns geätzte Platine, welche mit unseren Sensoren bestückt ist. Es gibt mehrere positive Aspekte, die eine eigene Platine mit sich bringt. Zum einen bietet sie eine stabile Plattform für die Befestigung der Sensoren. Zum anderen sparen wir uns dadurch eine Menge Kabel, welche deutlich störanfälliger sind als eine Platine. Die Platine hat an den entsprechenden Stellen Bohrungen, um sie mit der Zwischenwand und dem Mikrokontrollerboard zu verbinden. Die Platine bietet Platz für die folgenden Module:

\begin{itemize}
	\item BMP108 Drucksensor: Misst den Luftdruck und gibt diesen, sowie die daraus berechnete Höhe, zurück
	\item Sparkfun UV Sensor: Misst die Intensität der Strahlung des Spektrums 270-380 nm, welches dem UVA und UVB Spektrum entspricht
	\item TMP006 Infrarot Temperatursensor: Misst die Temperatur eines dünnen Aluminiumstückes in der Außenwand 
	\item Adafruit Ultimate GPS: Bestimmt die aktuelle Position, sowie die Höhe
	\item APC220 Transceiver Modul: Sendet die Daten als JSON-String zur Bodenstation
	\item Steckplatz zum Anschluss des Sharp-Feinstaubsensors: Misst den Anteil der Partikel, welche kleiner als 10 \textmu m sind
	\item Steckplatz zum Anschluss an das Mikrokontrollerboard: Bildet die Schnittstelle zwischen BeagleBone und Sensorikplatine
\end{itemize}



% Elektrisches Vorgehen, Schaltplan, Funkverbindung (Till)
\subsection{Elektrische Konstruktion}
Unser CanSat besteht aus mehreren Sensoren und einem zentralen Verarbeitungssystem sowie einem Sender. Diese kommunizieren alle über verschiedene Protokolle. Im Anhang unter der Einleitung befindet sich das Blockdiagramm unseres Satelliten. Im Blockdiagramm fehlen allerdings die verschiedenen Protokolle, in unserem Fall kommuniziert der BeagleBone Black, die MCU, mit allen Sensoren und holt deren Daten ab.

\begin{table}[H]
  \centering
    \begin{tabular}{rr}
    \toprule
    \textbf{Bauteil} & \textbf{Kommunikationsprotokoll} \\
    \midrule 
    UV ML8511 & ADC \\
    Sharp Feinstaubsensor & ADC \\
    APC220 & UART \\
    Ultimate GPS & UART \\
    TMP006 & I²C \\
    BMP180 & I²C \\
    \bottomrule
    \end{tabular}
    \caption{Kommunikationsprotokolle}
\end{table}

\subsubsection{Fachliche Grundlagen}
\paragraph{Embedded System}
Ein Embedded System ist in unserem Fall der BeagleBone Black. Das Mikrokontrollerboard taktet mithilfe eines ARM Cortex-A6 Prozessors mit 1Ghz. Auf ihm ist der leicht modifizierter Linux Kernel Angstrom mit Frontend installiert. Andere Beispiele für ein Embedded System sind etwa ein Smart TV oder ein Router. Beide besitzen eine Main Control Unit mit einem Betriebssystem, welches auf die Anwendung des Gerätes spezialisiert ist. In unserem Fall der BeagleBone, welcher verschiedene Technologien besitzt um mit einzelnen Bauteilen zu kommunizieren. UART, I-2-C, SPI, Analog, Digital, PWM, Timer, PRU, ADC, DAC und viele mehr. Viele dieser Technologien sind in unserem Projekt nicht in Verwendung. Jene, die in Verwendung sind, werden später im Dokument beschrieben.

\paragraph{Transistor-Transistor-Logik}
5V werden immer als logisch ``Ein'' bezeichnet. Damit ist gemeint, dass der Sensor, wenn er den höchsten Messwert erreicht, eine Spannung von 5V ausgibt. Ist dies nicht der Fall hat der Sensor eine andere Kennkurve die zum Beispiel bei 3.3V aufhört. Allgemein wird aber die Transistor-Transistor-Logik genutzt, welche 5V als logisch ``Ein'' und geerdet als logisch ``Aus'' ansieht. Es gibt natürlich Toleranzen, welche aber bei verschiedenen integrierte Schaltkreisen und Mikrokontrollern unterschiedlich sind.

\paragraph{Analog-to-Digital-Converter}
Andere Sensoren, beispielsweise der UV-Sensor, verfügen lediglich über einen internen Wiederstand. Der Wiederstand verändert sich in Abhängigkeit zu einer mathematischen Kurve. Dadurch entstehen unterschiedliche Spannungen, welche über den jeweiligen analogen Pin ausgegeben werden. Mithilfe eines Analog-to-Digital-Converter konvertieren wir das analoge Signal, zum Beispiel 5V, in das äquivalente digitale Signal mit einer Auflösung von 12 Bits. \\

\[
2^{12} \quad = 4096
\]

So können 4096 verschiedene Stufen dargestellt werden. Da ein analoges Signal theoretisch in unendlich viele Stufen unterteilt werden kann scheinen 4096 Stufen auf den ersten Blick nicht viel. Ein einzelner Schritt ist trotzdem im digitalen (12 Bits) sehr klein, was folgende Rechnung zeigt.

\[
\frac{5V}{4096} = 0.001220703125 V
\] \\

Das Ergebnis bedeutet, dass man mit 12 Bit eine 5V Spannung in 0.001220703125V Schritten darstellt kann. Dem Arduino Mega 2560 stehen nur 10 Bits zur Verfügung. Da die Rechnung exponentiell ist verkleinert sich dadurch die Anzahl der Stufen enorm.\\

\[
2^{10} \quad = 1024
\]

\[
\frac{5V}{1024} = 0.0048828125V
\] \\

Wie man an dem Ergebnis sieht, kann das BeagleBone den Wert, der am analogen Pin ankommt, viel genauer darstellen, als der Arduino. 

\paragraph{Universal-Asynchronous-Receiver-Transmitter}
UART ist eine digitale serielle Schnittstelle zum Realisieren von einfachen Kommunikationen zwischen zwei Endpunkten. Die Funktionsweise ist denkbar einfach. Wir nutzen in unserem Satelliten meist eine Baudrate von 9600bps. Baud ist die Schrittgeschwindigkeit oder Symbolrate, also 9600 bits per second. Für UART gibt es wie beim RJ45 Stecker TX und RX, die beim Aufbau einer Kommunikation gekreuzt werden. Dies liegt daran, dass der Transciever des einen Komponenten an den Reciever des anderen angeschlossen werden muss. Und natürlich auch andersherum. Nun wird zwischen vielen verschiedenen Arten von UART unterschieden in unserem Fall die TTL-UART Variante welche die beim Analog-to-Digital-Converter genannten 5V als logisch ``Ein'' bezeichnen. \\

\paragraph{Inter-Integrated-Circuit}
I-2-C ist ein serieller Datenbus der über zwei Kabel mit einer 10-Bit-Adressierung (welcher 1024 Stufen entspricht). Der Bus ist auf eine maximalen Geschwindigkeit von 5 Mbit/s beschränkt, welche für unsere Zwecke jedoch ausreichend ist. Der Sinn des Bussystems ist es, mithilfe von einer Adressen einen Datensatz oder Befehl nur an den gewünschten Empfänger zu senden. I-2-C benötigt lediglich eine Datenleitung, welche den eine Kommunikation zwischen dem Master (in unserem Fall das BeagleBone) und den Slaves (in unserem Fall die Sensoren) herstellt. Der Master kann über diese Datenleitung den Slaves sagen, wann welcher Slave seine Daten senden darf.

\subsubsection{Sensorik}
\paragraph{ML8511 - UV Sensor}
Der UV Sensor bietet im Inneren lediglich eine Fotodiode welche auf eine Wellenlänge zwischen 280 und 390 nm reagiert. Zusätzlich zu der Diode existiert ein Verstärker, welcher dafür sorgt, dass auch minimale Veränderung gemessen werden können. Die Fotodiode ändert je nach Einstrahlung von UV-A und -B ihren Wiederstand. Die dabei entstehende Veränderung in der Spannung ist messbar.

\begin{figure}[h]
	\centering
	\includegraphics[scale=0.4]{2_Beschreibung_des_CANSAT/graph_photodiode_response.png}
	\caption{Spannungsausgabe vs. UV Intensität}
	\label{graph photodiode}
\end{figure}


\paragraph{Sharp Feinstaubsensor}
Der Sharp Feinstaubsensor arbeitet, wie der ML8511, sehr simpel. Im Inneren befindet sich eine Infrarot Diode, welche die Partikel anstrahlt. Auf der anderen Seite befindet sich ein Fototransistor welcher dann feststellt, wie viel von diesem Licht von Partikeln reflektiert wird, diese Veränderung ist messbar.

\begin{figure}[h]
	\centering
	\includegraphics[scale=0.5]{2_Beschreibung_des_CANSAT/graph_photodiode_sharp.png}
	\caption{Spannungsausgabe vs. Staub}
	\label{graph photodiode}
\end{figure}

Um den Fototransistor nicht immer ganz zu bestrahlen, ist die Infrarot Diode nicht die gesamte Zeit angeschaltet. Sie scheint lediglich für den Bruchteil einer Sekunde. Die Messung muss innerhalb dieser Zeitspanne stattfinden.

\paragraph{APC220}
Der APC220, ist ein Transciever welcher entweder senden oder empfangen kann. Der BeagleBone Black schickt den formatierten JSON String per UART an den APC220. Dieser schickt ihn über die vorher am Computer festgelegte Frequenz in den Raum weiter. Am Boden befindet sich ebenfalls ein APC220, welcher die Daten empfängt.

\paragraph{Ultimate GPS}
Der Ultimate GPS von Adafruit verbindet sich mit den allgemeinen GPS-Satelliten und sendet Serial seine Daten im NMEA Format aus. Dieses Format gibt es in diversen Varianten, es verfügt aber in so gut wie allen Varianten über folgende Daten:
\begin{table}[H]
  \centering
    \begin{tabular}{rr}
    \toprule
    \textbf{Nummer} & \textbf{Daten} \\
    \midrule 
    1 & Breitengrad \\
    2 & Längengrad \\
    3 & Zeit \\
    4 & GPS-Qualität \\
    5 & Anzahl benutzter Satelliten \\
    6 & Höhe \\
    \bottomrule
    \end{tabular}
    \caption{Auslesebare Daten}
\end{table}

Außerdem können wir den Ultimate GPS über die serielle UART Schnittstelle konfigurieren um ihn zum Beispiel nur ein bestimmtes NMEA Format ausgeben zu lassen oder um die Bitrate der seriellen Übertragung zu ändern.

\paragraph{TMP006}
Der TMP006 ist ein Infrarot Temperatur Sensor, welcher die Temperatur von einem Objekt misst ohne in direktem Kontakt zu stehen. Der Sensor misst die Temperatur eines Objektes anhand der ausgestrahlten Energie auf Wellenlänge von 4 \textmu m bis zu 16 \textmu m. Durch die veränderte Spannung am Sensor ist eine Messung der Temperatur möglich. Je größer das Objekt ist, umso weiter entfernt muss es sich befinden, um vom Field of View des Sensors erfasst zu werden.

\begin{figure}[h]
	\centering
	\includegraphics[scale=0.5]{2_Beschreibung_des_CANSAT/sensor_fov.png}
	\caption{Sensor Field of View}
	\label{sensor fov}
\end{figure}

Die Messung kann sehr ungenau werden, da, je nach Außentemperatur und Temperatur der Sensorfläche selber, Fehler beim Messen entstehen können.

\paragraph{BMP180}
Der BMP180 ist ein Drucksensor welcher mithilfe einer Membran den Druck misst und diesen per On-Board Controller direkt in die Höhe umrechnet. Der Sensor gibt die Daten dann per I²C-Bus aus.

\subsubsection{Energieverbrauch}
\begin{table}[H]
  \centering
    \begin{tabular}{rrrl}
    \toprule
    \textbf{Bauteil} & \textbf{Stromaufnahme} & \textbf{Spannung} & \textbf{Leistungsaufnahme} \\
    \midrule
    Beaglebone Black  & 500mA & 5V & 2500mW \\
    ML8511& <1mA & 3.3V & <3.3mW \\
    TMP006& <1mA& 3.3V& <3.3mW \\
    Sharp& 20mA & 3.3V& 66mW\\
    BMP180& <1mA& 3.3V& <3.3mW \\
    Ultimate GPS Modul& 25mA&3.3V& 82.5mW \\
    APC220& 35mA & 5V & 175mW\\

    \bottomrule
     & & &2833.4mW \\
    \bottomrule
    \end{tabular}%
    \caption{Energieverbrauch}
  \label{tab:budgetausgaben}%
\end{table}%


% Programmiersprache, Entwicklungsumgebung, abschätzung Datenmenge, Programmablauf (PAP), Datenverarbeitung (Steffen)
\subsection{Softwaredesign}
\subsubsection{Python als Programmiersprache}
Als Programmiersprache für das BeagleBone Black haben wir uns für Python entschieden. Es wäre ebenfalls möglich gewesen, den Mikrocontroller mit den Sprachen JavaScript, Java, C, C++, C\# und vielen weiteren Sprachen zu programmieren. Da es sich bei dem BeagleBone um ein Linux-basiertes Embedded System handelt, unterstützt es praktisch alle Programmiersprachen, sofern entsprechende Bibliotheken existieren. Allerdings haben wir uns aufgrund der Tatsache, dass Python im Gegensatz zu Java nicht objektorientiert geschrieben werden muss, für Python entschieden. Wir möchten auf der Hardwareseite möglichst auf objektorientierte Programmierung verzichten. Ein weiteres wichtiges Argument ist die gute Python-Bibliothek, welche von einer großen Community permanent gewartet und aktualisiert wird.
\subsubsection{Datenverarbeitung auf dem BeagleBone}
Die Datenverarbeitung auf dem BeagleBone verläuft relativ simpel. Zunächst werden alle von den Sensoren aufgezeichneten Daten, bei Sensoren mit I2C-Anbindung, mithilfe von Libraries, und bei den anderen mithilfe von Umrechnungsalgorithmen, gesammelt. Anschließend werden alle gesammelten Daten in einen JSON-String geparsed, welcher mithilfe von unserem Transceiver an die Bodenstation übermittelt wird. Diese übernimmt hieraufhin die weitere Verarbeitung und Darstellung der Messdaten. Zusätzlich werden die Daten auf dem internen Speicher des BeagleBones gespeichert.


% Fallschirm (Alexander F.)
\subsection{Bergungssystem}
Für unseren Landesystem haben wir uns entschieden unseren eigenen Fallschirm zu bauen. Die Hauptaufgabe ist es, eine weiche Landung auf dem Boden zu garantieren. Die Vorgabe war, dass der Fallschirm und die Dose eine Fallgeschwindigkeit von 15 Meter/Sekunde haben sollen. Unsere Testfallschirme, die wir auch schon im Vorjahr genutzt haben, waren für eine deutlich geringer Fallgeschwindigkeit ausgelegt. Dieses Jahr wollen wir eine neue Fallschirmart testen. Die Idee dabei ist, dass die acht Schnüre, welche den Fallschirm an der Dose befestigt, mit sehr luftdurchlässigen Stoff, sogenannter Gaze, ersetzt werden. Wir erhoffen uns dadurch eine stabilere Lage in der Luft und ein fortschrittlicheres Designe.

\subsubsection{Berechnungen}
Für den Bau des Fallschirm wissen wir bereits, das dieser mit v = 15m/s fallen soll. Außerdem haben wir einen bereits berechneten Strömungswiderstandskoeffizient (Cw) von 1,33. Für die Berechnung des Fallschirms wurde folgende Formel verwendet:
\[
Fw = Cw*\frac{1}{2}*roh*v²*A
\]
Fw ist die Strömungswiderstandskraft. Diese kann ermittelt werden, indem man die Fallwiederstandskarft einsetzt.
\[
Fw = m * g = 250g * 9,81\frac{m}{s²} = 2,4525\frac{m}{s²}*Kg
\]
Um die Größe des Fallschirms zu berechnen, kann man nun durch Einsetzten in die erste Formel diese nach A umstellen.
\[
A=\frac{2*2,4525\frac{m}{s²}*Kg}{Cw*Roh*v²}
\]
\[
A=\frac{2*2,4525\frac{m}{s²}*Kg}{1,33*1,2\frac{Kg}{m³}*15²\frac{m²}{s²}} = 0,01365m² \text{ oder } 136,5cm²
\]
Eine Fläche von 136,5 cm² entspricht einem Durchmesser von ca. 34 cm.

\subsubsection{Bau}
Beim Bau der Fallschirme haben wir den Stoff von Regenschirmen verwendet. Dieser Stoff ist bereits in einer Art Halbkugelform mit acht aneinandergefügten Panels. Er hat genug Widerstand um den Belastung des Fluges stand zu halten und ist sogar regendicht. Zuerst muss das Gestell vorsichtig vom Stoff herunter geschnitten werden. Danach muss in der Mitte des Stoff, wo alle acht Kanten aufeinander Treffen, ein rund 5 cm großes Loch geschnitten werden. Dort kann die gestauchte Luft leicht entweichen, statt am Rand unkontrolliert auszutreten. Würde sie dies tun, so könnte es leicht zu gefährlichen Turbulenzen kommen. Nun kann der Fallschirm in die entsprechende Größe zugeschnitten werden. Am Rand des Fallschirms sollte ein zusätzlicher ca. 1 cm Rand gelassen werden, der in den Fallschirm umgeklappt wird. Dieser muss mit einer Nähmaschine, auf dem Fallschirm, fest genäht werden. Dies dient dazu, die Struktur des Randes besser zu schützen. Am Rand befindet sich die höchste Wahrscheinlichkeit das durch eine kleine Kerbe ein kompletter Riss entstehen könnte. Nun kann auf diese Grundlage die Gaze genäht werden. Beachtet werden sollte zusätzlich, das beim Übergang von Fallschirm und Dose am Befestigungspunkt eine hohe Belastung auftritt. An diesem Punkt hat die Gaze einen gewaltigen Nachteil gegenüber der Schnüre. Hier kann es durch eine einmalige starke Belastung zum Riss kommen, der sich im Laufe des Fluges weiter ausbreiten kann. Daher wurde dort eine Verstärkung mit einem sehr rissfestem Material eingenäht, der die Spannung erst kompensiert und diese dann gut auf die Gaze verteilt.

\subsection{Aufgetretene Probleme}
Während des Baus des CanSats sind selbstverständlich einige Probleme aufgetreten. Manche dieser Probleme waren relativ leicht zu lösen, andere erforderten eine detaillierte Recherche. Im Nachfolgenden werden einige dieser Probleme und unsere Lösungsansätze erläutert.

\begin{itemize}
	\item \textbf{Befestigung der Dosendeckel}: Es war vorerst geplant, durchgängige Gewindestangen zu verwenden, welche durch die Wand führen und so die Deckel und die Wand miteinander verbinden. Diese Stangen kollidierten jedoch mit den Löchern für die Befestigung des Mikrokontorollers und der Sensorikplatine. Nach verhältnismäßig langem Überlegen und Ausprobieren haben wir uns dazu entschlossen, die Gewindestangen nicht durchgängig zu gestalten. Stattdessen werden sie lediglich durch den Fuß und den Kopf der Wand gesteckt und dort verschraubt. Dies hat den Vorteil, dass wir das Gewicht deutlich verringern und wir innerhalb des CanSats wesentlich mehr Freiräume haben, um Objekte zu platzieren. Nachteilig ist jedoch, dass dadurch die Stabilität verringert wird.
	\item \textbf{Spektrum des UV-Sensors}: Unser UV -Sensor misst die Intensität der Strahlung, welche im Spektrum 280-390 nm liegt. Dies hat die Folge, dass der Output des Sensors deutlich über dem zu erwartenden Wert liegt und es relativ schwer fällt, einen Vergleich zwischen diversen Messungen aufzustellen. Dies liegt daran, dass man nicht verifizieren kann, welche Wellenlänge mit welchem Anteil an dem Gesamtoutput beteiligt sind.
	\item \textbf{Intensität des Sharp-Feinstaubsensors}: Zunächst war der Sharp-Sensor dazu gedacht, die Feinstaubkonzentration in unserer Athmosphäre zu messen. Jedoch mussten wir feststellen, dass der Sensor keineswegs Feinstäube, sondern groberen Staub, wie zum Beispiel Hausstaub, misst. Dies stellte deshalb ein Problem dar, da wir bis dato das komplette Dosendesign auf den Sharp-Sensor abstimmten. Ein neuer Sensor war zwar verfügbar, konnte jedoch aufgrund des Dosendesigns nicht integriert werden, da dieser zu groß ist. Da es durchaus möglich ist mit dem Sharp-Sensor halbwegs akzeptable Werte zu erlangen, wenn man Referenzmessungen durchführt, bleibt dieser zumindest vorerst im Gesamtsystem erhalten.
	\item \textbf{Geeigneter Ozon-Sensor}: Es hat sich als äußert schwierig erwiesen, einen Ozon Sensor zu finden, welcher keine lange Vorlaufzeit benötigt, verhältnismäßig klein ist und nicht übermäßig viel kostet. Daher ist es aktuell nicht mehr geplant, einen Ozon Sensor innerhalb des CanSats unterzubringen.
	\item \textbf{BeagleBone Black}: Zu Beginn unserer Arbeit am BeagleBone hatten wir einige Probleme mit diesem. Beispielsweise ließ sich zunächst die Python-Library für das BeagleBone nicht verwenden. Einige Pins ließen sich nicht ansteuern, was zu Fehlern führte. Letztlich fanden wir heraus, dass die Linux Distribution Debian auf dem System installiert war. Dieses ist jedoch für das BeagleBone Black noch in der Testphase und kann Bugs hervorrufen. Gelöst wurde das Problem, indem der Speicher mit dem Defaultsystem Linux Angstrom geflasht wurde. Danach funktionierte das Ausführen von Pythoncode ohne jegliche Probleme.
	\item \textbf{UART und I2C-Bus}: Bei der Übertragung mithilfe von UART und dem I²C-Bus sind wir auf Probleme in Kombination mit dem BeagleBone gestoßen. Dieses hatte verschiedene Ports und Protokolle standardmäßig nicht aktiviert. Wir mussten im Linux-System Angstrom einige Startroutinen hinzufügen, sodass bei jedem Start auch alle Ports und Protokolle aktiviert werden. Am Ende kostete dies viel Zeit, da die Dokumentation des BeagleBone in diesem Punkt nicht detailgenau war und für verschiedene Versionen verschiedene Lösungen des Problems existieren.
\item \textbf{Berechnung der Fallschirmgröße}: Das Ergebnis der Größenberechnung des Fallschirms erschien uns nicht realistisch. Der genutzte Cw-Wert hat einen Formfaktor, der nicht unseren Fallschirmen entsprach. Daher wurde beschlossen, einige Tests durchzuführen, um die Werte genauer zu bestimmen (siehe Testkonzept).
\item \textbf{Haltbarkeit des Gazestoffes}: Beim Bau des Fallschirms ist uns aufgefallen, dass der Stoff, der die Dose mit dem Fallschirm befestigt, eventuell nicht stark genug ist. Durch die sehr löchrige Struktur kann es schnell zu kleinen Rissen kommen. Werden diese zu stark belastet können diese sich ausbreiten.
\item \textbf{Durchmesser des CanSats}: Wir sind zu Beginn des Projektes davon ausgegangen, dass der Durchmesser des CanSats dem einer standardisierten Getränkedose entspricht (67mm). Zu spät ist uns aufgefallen, dass dies ein Irrtum war, und der Durchmesser 66mm betragen muss. Dies stellt jedoch kein Problem dar, da wir die Dicke der Außenwand problemlos um 0,5mm verringern können.
\end{itemize}

\subsection{Testkonzept}
Um sicherzustellen, dass der CanSat problemlos funktioniert, wurden diverse Tests durchgeführt. Dazu zählt natürlich das Prüfen auf Funktionstüchtigkeit der Sensoren. Hierfür wurde jeder Sensor separat an verschiedene Mikrocontroller (BeagleBone Black, Arduino Mega) angeschlossen. Dadurch konnte verifiziert werden, dass jeder Sensor unter jedem Board den gleichen Output liefert. Zusätzlich wurde überprüft, ob die Sensoren auf eine Veränderung der zu messenden Eigenschaft reagieren. Um zu erkennen, ob die gemessenen Werte den tatsächlichen Werten entsprechen, werden diese im Umweltlabor von \href{https://www.atlas-elektronik.com/atlas-elektronik/}{Atlas Elektronik} getestet und kalibriert. Dort soll ebenfalls überprüft werden, wie stabil der CanSat ist, um vorherzusagen, ob er beim Aufschlag beschädigt wird. Das Testkonzept für die Sensoren beruht im Wesentlichen auf Trial and Error.

\subsubsection{Test der Fallschirmgröße}
Um den Fallschirm mit einer zweiten Berechnung zu Prüfen, haben wir uns entschieden, die Zugkraft des Fallschirms zu messen. Aus einem Auto, welches mit einer Geschwindigkeit von 50 km/h fuhr, haben wir einige Testfallschirme an einem Newtonmeter befestigt und diesen währendes heraus gehalten. Aus der Geschwindigkeit und dem Widerstand des Newtonmeter konnten ermittelt werden, dass der Fallschirm keine 16 cm Durchmesser benötigt. Weitere Tests haben ergeben, dass 30 cm ein besserer Wert ist. Bei einem Fallschirm dieser Größe können auch schon relative kleine Veränderung der Größe die Fallgeschwindigkeit enorm erhöhen oder senken.