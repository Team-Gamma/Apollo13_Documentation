\subsection{Bergungssystem}
Für unseren Ladensystem haben wir uns entschieden unseren eigenen Fallschirm zu bauen. Die Hauptaufgabe ist es, was einen Fallschirm ausmacht, eine weiche Landung auf dem Boden. Was es für uns schwierig gemacht hat, war die gestellte Fallgeschwindigkeit von 15 Metern/Sekunde. Unsere Testfallschirme, die wir auch schon im Vorjahr genutzt haben, waren für eine deutlich geringer Fallgeschwindigkeit ausgelegt. Dieses Jahr wollen wir in CanSat 2015 eine neue Art eines Fallschirms testen. Wir haben die normalen 8 Schnüre zum Befestigen des Fallschirms mit der Dose mit sehr Luft durchlässigen Stoff ersetzt. Wir erhoffen uns dadurch eine stabilere Lage in der Luft und ein fortschrittlicheres Designe.

	\subsubsection{Bau}
	Beim Bau der Fallschirme haben wir Regenschirme benutzt. Diese sind bereits in einer Art Halbkugelform mit acht aneinandergefügten Panels. Dieser Stoff hat genug Wiederstand um der Belastung eines CanSat stand zu halten und ist sogar regendicht. Zuerst muss das Gestell vorsichtig vom Stoff herunter geschnitten werden. Danach wird in der Mitte des Stoffes, wo alle 8 Kanten aufeinander Treffen ein ungefähr 5 cm großes Loch geschnitten. Nun kann der Fallschirm-Rohling in die entsprechende Größe geschnitten werden. Am Rand des Fallschirms sollte ein zusätzlicher 1 cm Rand gelassen werden, der in den Fallschirm umgeklappt wird. Dieser muss dann mit einer Nähmaschine vergenäht werden. Dies dient dazu die Struktur des Randes, durch Falten in den Innenbereichs besser vor stärkeren Belastung zu schützen. Am Rand befindet sich die höchste Wahrscheinlichkeit das durch eine kleine Kerbe ein kompletter Riss entstehen könnte. Nun kann auf dieser Grundlage der sehr luftdurchlässige Stoff drauf genäht werden. Durch zu hohe Belastungen am Angelpunkt mit dem der Dose sollte dort ein Stoff flicken drauf genäht werden, um die Belastung auf alle Seiten gleichmäßig zu verteilen.

	\subsubsection{Messungen}
	Um den Fallschirm auf die richtige Größe zu bekommen, haben wir eine Reihe von Test gemacht. Für uns waren die Berechnungen der Fallschirm Größe zu einfach und ungenau, deshalb haben wir nach alternativen Möglichkeiten gesucht. Unsere neue Idee war, die Größe mit Newton-Messer zu bestimmen. Der Fallschirm muss für die 15 Meter/Sekunde mit dem Gewicht von ca 250 Gramm eine Newton Wert von 12 haben. Die Messungen haben wir mit einigen Prototypen-Fallschirmen aus einem fahrenden Auto gemacht. Bei einer Geschwindigkeit von 50Km/h sollte die Prototypen diesen Wert so nah wie möglich kommen. Durch mehrere Fallschirmtest kann dadurch die Größe auf praktisch einen Meter genau bestimmt werden.  