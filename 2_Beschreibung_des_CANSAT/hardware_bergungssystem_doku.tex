\subsection{Bergungssystem}
Für unseren Landesystem haben wir uns entschieden unseren eigenen Fallschirm zu bauen. Die Hauptaufgabe ist es, eine weiche Landung auf dem Boden zu garantieren. Die Vorgabe war, dass der Fallschirm und die Dose eine Fallgeschwindigkeit von 15 Meter/Sekunde haben sollen. Unsere Testfallschirme, die wir auch schon im Vorjahr genutzt haben, waren für eine deutlich geringer Fallgeschwindigkeit ausgelegt. Dieses Jahr wollen wir eine neue Fallschirmart testen. Die Idee dabei ist, dass die acht Schnüre, welche den Fallschirm an der Dose befestigt, mit sehr luftdurchlässigen Stoff, sogenannter Gaze, ersetzt werden. Wir erhoffen uns dadurch eine stabilere Lage in der Luft und ein fortschrittlicheres Designe.

\subsubsection{Berechnungen}
Für den Bau des Fallschirm wissen wir bereits, das dieser mit v = 15m/s fallen soll. Außerdem haben wir einen bereits berechneten Strömungswiderstandskoeffizient (Cw) von 1,33. Für die Berechnung des Fallschirms wurde folgende Formel verwendet:
\[
Fw = Cw*\frac{1}{2}*roh*v²*A
\]
Fw ist die Strömungswiderstandskraft. Diese kann ermittelt werden, indem man die Fallwiederstandskarft einsetzt.
\[
Fw = m * g = 250g * 9,81\frac{m}{s²} = 2,4525\frac{m}{s²}*Kg
\]
Um die Größe des Fallschirms zu berechnen, kann man nun durch Einsetzten in die erste Formel diese nach A umstellen.
\[
A=\frac{2*2,4525\frac{m}{s²}*Kg}{Cw*Roh*v²}
\]
\[
A=\frac{2*2,4525\frac{m}{s²}*Kg}{1,33*1,2\frac{Kg}{m³}*15²\frac{m²}{s²}} = 0,01365m² \text{ oder } 136,5cm²
\]
Eine Fläche von 136,5 cm² entspricht einem Durchmesser von ca. 34 cm.

\subsubsection{Bau}
Beim Bau der Fallschirme haben wir den Stoff von Regenschirmen verwendet. Dieser Stoff ist bereits in einer Art Halbkugelform mit acht aneinandergefügten Panels. Er hat genug Widerstand um den Belastung des Fluges stand zu halten und ist sogar regendicht. Zuerst muss das Gestell vorsichtig vom Stoff herunter geschnitten werden. Danach muss in der Mitte des Stoff, wo alle acht Kanten aufeinander Treffen, ein rund 5 cm großes Loch geschnitten werden. Dort kann die gestauchte Luft leicht entweichen, statt am Rand unkontrolliert auszutreten. Würde sie dies tun, so könnte es leicht zu gefährlichen Turbulenzen kommen. Nun kann der Fallschirm in die entsprechende Größe zugeschnitten werden. Am Rand des Fallschirms sollte ein zusätzlicher ca. 1 cm Rand gelassen werden, der in den Fallschirm umgeklappt wird. Dieser muss mit einer Nähmaschine, auf dem Fallschirm, fest genäht werden. Dies dient dazu, die Struktur des Randes besser zu schützen. Am Rand befindet sich die höchste Wahrscheinlichkeit das durch eine kleine Kerbe ein kompletter Riss entstehen könnte. Nun kann auf diese Grundlage die Gaze genäht werden. Beachtet werden sollte zusätzlich, das beim Übergang von Fallschirm und Dose am Befestigungspunkt eine hohe Belastung auftritt. An diesem Punkt hat die Gaze einen gewaltigen Nachteil gegenüber der Schnüre. Hier kann es durch eine einmalige starke Belastung zum Riss kommen, der sich im Laufe des Fluges weiter ausbreiten kann. Daher wurde dort eine Verstärkung mit einem sehr rissfestem Material eingenäht, der die Spannung erst kompensiert und diese dann gut auf die Gaze verteilt.