\documentclass{article}
\usepackage{pgffor}
\usepackage{filecontents}
\usepackage{pdfpages }
\usepackage{graphicx} 
\usepackage{a4wide}
\usepackage{here} 
\usepackage{tabularx}
\usepackage[colorlinks]{hyperref}
\usepackage[german]{babel}
\usepackage[utf8]{inputenc}
\usepackage[T1]{fontenc}
\usepackage[3D]{movie15}
\usepackage{booktabs} 
\usepackage{eurosym}
\usepackage{verbatim}
\usepackage{amsmath}
\usepackage{textcomp}
\usepackage{caption}
\usepackage{setspace} 
\usepackage{geometry}
\geometry{a4paper, top=25mm}

\setcounter{tocdepth}{5}  % Ebenen für Aufnahme in das Inhaltsverzeichnis
\setcounter{secnumdepth}{4}  % Ebenen für Nummerierung


\begin{document}


% Titelseite
\title{CanSat 2015 Team Gamma Dokumentation}
\date{\today}
\author{Alexander Brennecke \and Till Schlechtweg \and Marc Huisinga \and Robin Bley \and Steffen Wißmann \and Alexander Feldmann \and Kevin Neumeyer}
\clearpage
\maketitle
\thispagestyle{empty}
\newpage
\thispagestyle{empty}
\tableofcontents
\thispagestyle{empty}
\newpage
\begin{abstract}
Diese Dokumentation dokumentiert die Arbeit an dem Dosensatelliten ``Apollo 13'' und die Entwicklung einer Bodenstation zur Datenverarbeitung der Daten des Satelliten im Rahmen des Deutschen CanSat-Wettbewerbs 2015 und des P5-Projekts der Europaschule SZII Utbremen.
\end{abstract}
\thispagestyle{empty}
\newpage
\setcounter{page}{1}

% Import der Einleitung
\section{Einleitung}
\subsection{Das Team}

Das gesamte Team besteht aus sieben Schülern und zwei betreuenden Lehrern. Die sieben Schüler sind jedoch intern in mehrere kleinere Teams aufgeteilt. Innerhalb der Teams ist jedoch kein Teammitglied vollkommen an seine Aufgaben gebunden, da uns ein guter Austausch und eine hervorragende Zusammenarbeit zwischen den einzeln Teammitgliedern und Teams wichtig ist. Die Arbeit der Gruppen und der einzelnen Personen werden im folgenden erläutert:

\begin{itemize}
\item Das Hardware Team besteht aus drei Personen, welche sich um den Bau des Satelliten selber, dem Design und dem Bau der Dose sowie der Programmierung des Mikrocontrollers kümmern. Zu diesem Team zählen folgende Personen:
\begin {description}
\item Alexander Brennecke ist verantwortlich für das Design der Dose. Dazu zählt die Konstruktion der eigentlichen Dose und die Anordnung der Sensoren im inneren der Dose.

\item Till Schlechtweg ist verantwortlich für die Funktionalität des Mikrocontrollers und den ausgewählten Sensoren.

\item Steffen Wißmann ist verantwortlich für die Übertragung der Daten zur Bodenstation und dem Programmcode des Mikrocontrollers.
\end {description}
\item Das Software Team besteht aus vier Personen, welche sich um das Programmieren des Analysetools und der Android Applikation kümmern. Dieses Team besteht aus folgenden Personen:
\begin {description}
\item Robin Bley

\item Alexander Feldmann

\item Marc Huisinga

\item Kevin Neumeyer
\end {description}
\item Zudem gib es ein Team, bestehend aus Alexander Brennecke und Till Schlechtweg, zur Organisation, Kommunikation mit Sponsoren und Öffentlichkeitsarbeit.

\item Betreut wird das Projekt durch zwei Lehrer unserer Schule:
\begin{description}
\item Mathematiklehrer Harm Hörnlein
\item Physiklehrer Frank Marshall
\end{description}

\end{itemize}

Die Arbeit an dem Projekt findet zum größten Teil wöchentlich am Dienstag und Mittwoch Nachmittag in den Laboren unserer Schule statt. Die Labore sind mit diversen Werkzeugen ausgestattet, sodass sowohl die Software als auch die Hardware Gruppe dort problemlos arbeiten kann. Zusätzlich zu diesen vier bis acht Stunden pro Woche kommen fünf Projekttage, welche uns von der Schule gestellt wurden. Aber natürlich arbeitet jedes Teammitglied auch außerhalb dieser Treffen an seinem Fachgebiet, soweit dies möglich ist. Zusätzlich gibt es immer wieder Treffen mit externen Unterstützungen, oder Zeit in der Schule, wenn Vertretungs- oder Mitbetreuungsunterricht stattfindet.

\subsection{Stärken des Teams}
Die große Stärke des Teams ist es, dass es auch schon vor diesem Projekt existiert hat und sich somit sehr gut kennt. Ebenfalls von Vorteil ist, dass jedes Teammitglied durch unsere schulische Ausbildung genügend Wissen hat, um auch außerhalb seines Fachgebietes unterstützend tätig zu sein.
\subsection{Verbesserungsbereiche des Teams}

\subsection{Das Missionsziel}

Die Idee hinter dem gesamten Projekts bezieht sich auf die extremen Umweltbelastung und ihre Folgen für den Menschlichen Körper. Ausschlaggebend für diese Idee ist ein Zeitungsartikel der Zeit, welcher über eine drohende Klage der EU-Kommission in Brüssel berichtet. (vgl. Die Zeit, 24.10.2014). Die EU-Kommission droht mit einer Klage gegen Deutschland, da die deutsche Bundesregierung bisher zu wenig Aufwand betreibt, um die Feinstaubkonzentration in der Luft zu reduzieren. Wir möchten diesen Aspekt aufgreifen und Messungen durchführen um die tatsächlichen Werte zu bestimmen. Der CanSat Wettbewerb eignet sich optimal dazu, da er uns die Möglichkeit bietet die Messungen nicht nur auf dem Boden sondern in verschiedenen Schichten der Atmosphäre durchzuführen. Feinstäube stehen in Verdacht, Krankheiten wie Asthma, Herz-Kreislauf Beschwerden und Krebs zu begünstigen.

Da der menschliche Körper nicht nur durch Feinstaub belastet wird haben wir uns entschlossen auch die Intensität der UV-Strahlung, welche die Hauptursache für Hautkrebserkrankungen ist, zu messen. Zusätzlich soll auch der Ozonwert bestimmt werden, da Ozon bereits in geringen Konzentrationen gesundheitsschädlich ist und zu Reizungen der Atemwege führen kann.

Für sich genommen ist jede dieser drei Größen schädlich für den Menschen. Im Zuge des Projektes wollen wir jedoch versuchen herauszufinden, ob es einen Zusammenhang zwischen ihnen gibt. Beispielsweise ist herauszufinden, ob ein höherer Ozon Gehalt gleichzeitig einen niedrigeren Feinstaubgehalt mit sich bringt.

Zusätzlich zum Bau des Messsystems im CanSat ist es unser Ziel eine einwandfreie Verarbeitung, Analyse und Präsentation der gemessenen Werte zu erzielen. Um dies zu garantieren programmieren wir ein eigenes Analysetool. Dieses Tool ermöglicht es uns die gemessenen Werte, während des Fluges des Satelliten, auszuwerten. Die Werte sollen dabei anschaulich und in Abhängigkeit zueinander dargestellt werden.

Um die Daten auch mobil verfügbar zu haben wollen wir eine Android Applikation bereitstellen. Diese Applikation soll vorerst nur für unser Projekt optimiert sein, bei Erfolg jedoch auch die Werte andere Teams anzeigen können.

\subsection{Praktischer Nutzen für den Auftragsgeber}





% Import der Hardware Doku
\section{Beschreibung des CANSAT}


% Blockdiagramm (irgendjemand)
\subsection{Missionsüberblick}
Wir haben uns für den Satelliten überlegt, dass dieser so weit wie möglich individuell sein sollte. Daher greifen wir nicht auf das, vom Wettbewerb bereitgestellte T-Minus CanSat Kit zurück. Stattdessen haben wir uns im Detail überlegt, welche Sensoren unseren Erwartungen entsprechen und wie wir diese bestmöglich innerhalb der Dose platzieren können. Zusätzlich möchten wir nicht auf eine Cola-Dose als Hülle zurück greifen, sondern möchten auch hier unser eigenes Design erschaffen.

% 3D Skizze, Erklärung der einzelnen Bestandteile (Alexander B.)
\subsection{Mechanik- und Strukturdesign}

Wir haben den CanSat in drei Komponenten aufgeteilt: Die Hülle, die Innenwand und die Sensorikplatine. Diese drei Komponenten bilden den Hauptbestandteil des CanSats und haben maßgeblich zu dem mechanischen und strukturellem Design beigetragen. Im nachfolgenden wird kurz auf jeden dieser Komponenten eingegangen und die exakte Funktion im Zusammenhang erklärt.

\subsubsection{Fachliche Grundlagen}
Um die 3D-gedruckte Wand zu erzeugen, wurde die 3D-Moddelierungssoftware \href{http://www.sketchup.com/de} {Sketchup} von Google verwendet. Sketchup bietet die Möglichkeit, vergleichsweise einfach 3D-Modelle zu zeichnen. Um dies zu tun, muss klar sein, welche Objekte gezeichnet werden sollen. Diese Objekte müssen vermessen und innerhalb von Sketchup gezeichnet werden. Dies erfordert die Kenntnis über gewisse mathematische Methoden zur Berechnung von Kreisen, Flächen und Körpern. Die meisten 3D-Drucker benötigen Dateien des Typs .stl, welche in Sketchup mit einem Plugin erzeugt werden können.
Zum Anfertigen von GFK-Komponenten wird ein Körper benötigt, auf welchen das GFK laminiert werden kann. In unserem Fall ist dieser Körper zylindrisch, mit einem Durchmesser von 31,5 mm, und aus Aluminium gefräst. \\
Um die Platine zu erstellen, wurde die Design-Software \href{http://www.cadsoft.de/eagle-pcb-design-software/} {Eagle PCB} verwendet. Eagle bietet die Möglichkeit, sowohl Schaltpläne als auch das entsprechende Layout zu erstellen. Im Anschluss wurde die Platine, mit Hilfe und Mitteln des \href{https://www.hackerspace-bremen.de/}{Hackerspace Bremen e.V.}, geätzt.

\subsubsection{Hülle}
Wir haben uns dazu entschieden, die äußere Hülle aus GFK (Glasfaser verstärkter Kunststoff) anzufertigen. Dieser hat die Eigenschaft, dass er bei einem sehr geringen Gewicht, und bei einer geringen Wandstärke, trotzdem eine gewisse Stabilität aufweist. Aus dem GFK haben wir eine Röhre mit einem Innendurchmesser von 31,5 mm und einem Außendurchmesser von 33,5 mm laminiert. Diese Röhre wurde auf eine Länge von 111 mm gekürzt und gefeilt. Um die Röhre oben und unten zu verschließen, haben wir uns bei \href{http://www.thyssenkrupp-system-engineering.com/de/home.html}{Thyssen Krupp System Engineering} zwei Aluminiumdeckel fräsen lassen. Diese haben uns ebenfalls durch ihr geringes Gewicht und ihre hohe Stabilität überzeugt. Die Deckel haben, genauso wie die Hülle, einen Außendurchmesser von 33,5 mm. Sie sind beide 2 mm dick und haben zusätzlich eine 2mm dicke Erhöhung, welche in die Röhre eingelassen wird.

\subsubsection{Innenwand}
Um die Elektronik innerhalb der Hülle zu platzieren und zu befestigen, haben wir uns dazu entschieden, eine Wand anzufertigen. Diese Wand teilt die Hülle mittig und bietet so auf beiden Seiten Platz, um unser Mikrokontrollerboard und unsere Sensorikplatine zu befestigen. Beide Bauteile werden mittels vier Gewindestangen an der Wand befestigt. Durch die Technik des 3D-Druckens ist es möglich, der Wand ein sehr geringes Gewicht bei einer verhältnismäßig hohen Stabilität zu verleihen. Zusätzlich gibt es uns die Möglichkeit, die Wand millimetergenau zu gestalten. \\
Am unteren Ende der Wand befindet sich eine Aushöhlung, sowie ein Fuß. Diese ist zum einen dafür da, um den Sharp-Feinstaubsensor zu befestigen. Zum anderen gibt der Fuß der Wand, und somit dem gesamten Satelliten, eine gewisse Stabilität. Der Fuß besitzt auf der einen Seite der Wand Bohrungen. Diese Bohrungen werden verwendet, um die Aluminiumdeckel an der Wand zu befestigen. An der oberen Seite der Wand befinden sich ebenfalls solche Bohrungen, um den oberen Deckel der Hülle zu befestigen. Da der Feinstaubsensor einen Luftzug benötigt, gibt es eine Bohrung, welche vertikal durch die Wand führt. Um das Mikrokontrollerboard mit der Sensorikplatine zu verbinden, existiert ein Fenster in der Mitte der Wand. Um die Sensorikplatine und das Mikrokontrollerboard an der Wand zu befestigen, existieren in der Wand weitere vier Bohrungen.

\subsubsection{Sensorikplatine}
Die Sensorikplatine ist eine von uns geätzte Platine, welche mit unseren Sensoren bestückt ist. Es gibt mehrere positive Aspekte, die eine eigene Platine mit sich bringt. Zum einen bietet sie eine stabile Plattform für die Befestigung der Sensoren. Zum anderen sparen wir uns dadurch eine Menge Kabel, welche deutlich störanfälliger sind als eine Platine. Die Platine hat an den entsprechenden Stellen Bohrungen, um sie mit der Zwischenwand und dem Mikrokontrollerboard zu verbinden. Die Platine bietet Platz für die folgenden Module:

\begin{itemize}
	\item BMP108 Drucksensor: Misst den Luftdruck und gibt diesen, sowie die daraus berechnete Höhe, zurück
	\item Sparkfun UV Sensor: Misst die Intensität der Strahlung des Spektrums 270-380 nm, welches dem UVA und UVB Spektrum entspricht
	\item TMP006 Infrarot Temperatursensor: Misst die Temperatur eines dünnen Aluminiumstückes in der Außenwand 
	\item Adafruit Ultimate GPS: Bestimmt die aktuelle Position, sowie die Höhe
	\item APC220 Transceiver Modul: Sendet die Daten als JSON-String zur Bodenstation
	\item Steckplatz zum Anschluss des Sharp-Feinstaubsensors: Misst den Anteil der Partikel, welche kleiner als 10 \textmu m sind
	\item Steckplatz zum Anschluss an das Mikrokontrollerboard: Bildet die Schnittstelle zwischen BeagleBone und Sensorikplatine
\end{itemize}



% Elektrisches Vorgehen, Schaltplan, Funkverbindung (Till)
\subsection{Elektrische Konstruktion}
Unser CanSat besteht aus mehreren Sensoren und einem zentralen Verarbeitungssystem sowie einem Sender. Diese kommunizieren alle über verschiedene Protokolle. Im Anhang unter der Einleitung befindet sich das Blockdiagramm unseres Satelliten. Im Blockdiagramm fehlen allerdings die verschiedenen Protokolle, in unserem Fall kommuniziert der BeagleBone Black, die MCU, mit allen Sensoren und holt deren Daten ab.

\begin{table}[H]
  \centering
    \begin{tabular}{rr}
    \toprule
    \textbf{Bauteil} & \textbf{Kommunikationsprotokoll} \\
    \midrule 
    UV ML8511 & ADC \\
    Sharp Feinstaubsensor & ADC \\
    APC220 & UART \\
    Ultimate GPS & UART \\
    TMP006 & I²C \\
    BMP180 & I²C \\
    \bottomrule
    \end{tabular}
    \caption{Kommunikationsprotokolle}
\end{table}

\subsubsection{Fachliche Grundlagen}
\paragraph{Embedded System}
Ein Embedded System ist in unserem Fall der BeagleBone Black. Das Mikrokontrollerboard taktet mithilfe eines ARM Cortex-A6 Prozessors mit 1Ghz. Auf ihm ist der leicht modifizierter Linux Kernel Angstrom mit Frontend installiert. Andere Beispiele für ein Embedded System sind etwa ein Smart TV oder ein Router. Beide besitzen eine Main Control Unit mit einem Betriebssystem, welches auf die Anwendung des Gerätes spezialisiert ist. In unserem Fall der BeagleBone, welcher verschiedene Technologien besitzt um mit einzelnen Bauteilen zu kommunizieren. UART, I-2-C, SPI, Analog, Digital, PWM, Timer, PRU, ADC, DAC und viele mehr. Viele dieser Technologien sind in unserem Projekt nicht in Verwendung. Jene, die in Verwendung sind, werden später im Dokument beschrieben.

\paragraph{Transistor-Transistor-Logik}
5V werden immer als logisch ``Ein'' bezeichnet. Damit ist gemeint, dass der Sensor, wenn er den höchsten Messwert erreicht, eine Spannung von 5V ausgibt. Ist dies nicht der Fall hat der Sensor eine andere Kennkurve die zum Beispiel bei 3.3V aufhört. Allgemein wird aber die Transistor-Transistor-Logik genutzt, welche 5V als logisch ``Ein'' und geerdet als logisch ``Aus'' ansieht. Es gibt natürlich Toleranzen, welche aber bei verschiedenen integrierte Schaltkreisen und Mikrokontrollern unterschiedlich sind.

\paragraph{Analog-to-Digital-Converter}
Andere Sensoren, beispielsweise der UV-Sensor, verfügen lediglich über einen internen Wiederstand. Der Wiederstand verändert sich in Abhängigkeit zu einer mathematischen Kurve. Dadurch entstehen unterschiedliche Spannungen, welche über den jeweiligen analogen Pin ausgegeben werden. Mithilfe eines Analog-to-Digital-Converter konvertieren wir das analoge Signal, zum Beispiel 5V, in das äquivalente digitale Signal mit einer Auflösung von 12 Bits. \\

\[
2^{12} \quad = 4096
\]

So können 4096 verschiedene Stufen dargestellt werden. Da ein analoges Signal theoretisch in unendlich viele Stufen unterteilt werden kann scheinen 4096 Stufen auf den ersten Blick nicht viel. Ein einzelner Schritt ist trotzdem im digitalen (12 Bits) sehr klein, was folgende Rechnung zeigt.

\[
\frac{5V}{4096} = 0.001220703125 V
\] \\

Das Ergebnis bedeutet, dass man mit 12 Bit eine 5V Spannung in 0.001220703125V Schritten darstellt kann. Dem Arduino Mega 2560 stehen nur 10 Bits zur Verfügung. Da die Rechnung exponentiell ist verkleinert sich dadurch die Anzahl der Stufen enorm.\\

\[
2^{10} \quad = 1024
\]

\[
\frac{5V}{1024} = 0.0048828125V
\] \\

Wie man an dem Ergebnis sieht, kann das BeagleBone den Wert, der am analogen Pin ankommt, viel genauer darstellen, als der Arduino. 

\paragraph{Universal-Asynchronous-Receiver-Transmitter}
UART ist eine digitale serielle Schnittstelle zum Realisieren von einfachen Kommunikationen zwischen zwei Endpunkten. Die Funktionsweise ist denkbar einfach. Wir nutzen in unserem Satelliten meist eine Baudrate von 9600bps. Baud ist die Schrittgeschwindigkeit oder Symbolrate, also 9600 bits per second. Für UART gibt es wie beim RJ45 Stecker TX und RX, die beim Aufbau einer Kommunikation gekreuzt werden. Dies liegt daran, dass der Transciever des einen Komponenten an den Reciever des anderen angeschlossen werden muss. Und natürlich auch andersherum. Nun wird zwischen vielen verschiedenen Arten von UART unterschieden in unserem Fall die TTL-UART Variante welche die beim Analog-to-Digital-Converter genannten 5V als logisch ``Ein'' bezeichnen. \\

\paragraph{Inter-Integrated-Circuit}
I-2-C ist ein serieller Datenbus der über zwei Kabel mit einer 10-Bit-Adressierung (welcher 1024 Stufen entspricht). Der Bus ist auf eine maximalen Geschwindigkeit von 5 Mbit/s beschränkt, welche für unsere Zwecke jedoch ausreichend ist. Der Sinn des Bussystems ist es, mithilfe von einer Adressen einen Datensatz oder Befehl nur an den gewünschten Empfänger zu senden. I-2-C benötigt lediglich eine Datenleitung, welche den eine Kommunikation zwischen dem Master (in unserem Fall das BeagleBone) und den Slaves (in unserem Fall die Sensoren) herstellt. Der Master kann über diese Datenleitung den Slaves sagen, wann welcher Slave seine Daten senden darf.

\subsubsection{Sensorik}
\paragraph{ML8511 - UV Sensor}
Der UV Sensor bietet im Inneren lediglich eine Fotodiode welche auf eine Wellenlänge zwischen 280 und 390 nm reagiert. Zusätzlich zu der Diode existiert ein Verstärker, welcher dafür sorgt, dass auch minimale Veränderung gemessen werden können. Die Fotodiode ändert je nach Einstrahlung von UV-A und -B ihren Wiederstand. Die dabei entstehende Veränderung in der Spannung ist messbar.

\begin{figure}[h]
	\centering
	\includegraphics[scale=0.4]{2_Beschreibung_des_CANSAT/graph_photodiode_response.png}
	\caption{Spannungsausgabe vs. UV Intensität}
	\label{graph photodiode}
\end{figure}


\paragraph{Sharp Feinstaubsensor}
Der Sharp Feinstaubsensor arbeitet, wie der ML8511, sehr simpel. Im Inneren befindet sich eine Infrarot Diode, welche die Partikel anstrahlt. Auf der anderen Seite befindet sich ein Fototransistor welcher dann feststellt, wie viel von diesem Licht von Partikeln reflektiert wird, diese Veränderung ist messbar.

\begin{figure}[h]
	\centering
	\includegraphics[scale=0.5]{2_Beschreibung_des_CANSAT/graph_photodiode_sharp.png}
	\caption{Spannungsausgabe vs. Staub}
	\label{graph photodiode}
\end{figure}

Um den Fototransistor nicht immer ganz zu bestrahlen, ist die Infrarot Diode nicht die gesamte Zeit angeschaltet. Sie scheint lediglich für den Bruchteil einer Sekunde. Die Messung muss innerhalb dieser Zeitspanne stattfinden.

\paragraph{APC220}
Der APC220, ist ein Transciever welcher entweder senden oder empfangen kann. Der BeagleBone Black schickt den formatierten JSON String per UART an den APC220. Dieser schickt ihn über die vorher am Computer festgelegte Frequenz in den Raum weiter. Am Boden befindet sich ebenfalls ein APC220, welcher die Daten empfängt.

\paragraph{Ultimate GPS}
Der Ultimate GPS von Adafruit verbindet sich mit den allgemeinen GPS-Satelliten und sendet Serial seine Daten im NMEA Format aus. Dieses Format gibt es in diversen Varianten, es verfügt aber in so gut wie allen Varianten über folgende Daten:
\begin{table}[H]
  \centering
    \begin{tabular}{rr}
    \toprule
    \textbf{Nummer} & \textbf{Daten} \\
    \midrule 
    1 & Breitengrad \\
    2 & Längengrad \\
    3 & Zeit \\
    4 & GPS-Qualität \\
    5 & Anzahl benutzter Satelliten \\
    6 & Höhe \\
    \bottomrule
    \end{tabular}
    \caption{Auslesebare Daten}
\end{table}

Außerdem können wir den Ultimate GPS über die serielle UART Schnittstelle konfigurieren um ihn zum Beispiel nur ein bestimmtes NMEA Format ausgeben zu lassen oder um die Bitrate der seriellen Übertragung zu ändern.

\paragraph{TMP006}
Der TMP006 ist ein Infrarot Temperatur Sensor, welcher die Temperatur von einem Objekt misst ohne in direktem Kontakt zu stehen. Der Sensor misst die Temperatur eines Objektes anhand der ausgestrahlten Energie auf Wellenlänge von 4 \textmu m bis zu 16 \textmu m. Durch die veränderte Spannung am Sensor ist eine Messung der Temperatur möglich. Je größer das Objekt ist, umso weiter entfernt muss es sich befinden, um vom Field of View des Sensors erfasst zu werden.

\begin{figure}[h]
	\centering
	\includegraphics[scale=0.5]{2_Beschreibung_des_CANSAT/sensor_fov.png}
	\caption{Sensor Field of View}
	\label{sensor fov}
\end{figure}

Die Messung kann sehr ungenau werden, da, je nach Außentemperatur und Temperatur der Sensorfläche selber, Fehler beim Messen entstehen können.

\paragraph{BMP180}
Der BMP180 ist ein Drucksensor welcher mithilfe einer Membran den Druck misst und diesen per On-Board Controller direkt in die Höhe umrechnet. Der Sensor gibt die Daten dann per I²C-Bus aus.

\subsubsection{Energieverbrauch}
\begin{table}[H]
  \centering
    \begin{tabular}{rrrl}
    \toprule
    \textbf{Bauteil} & \textbf{Stromaufnahme} & \textbf{Spannung} & \textbf{Leistungsaufnahme} \\
    \midrule
    Beaglebone Black  & 500mA & 5V & 2500mW \\
    ML8511& <1mA & 3.3V & <3.3mW \\
    TMP006& <1mA& 3.3V& <3.3mW \\
    Sharp& 20mA & 3.3V& 66mW\\
    BMP180& <1mA& 3.3V& <3.3mW \\
    Ultimate GPS Modul& 25mA&3.3V& 82.5mW \\
    APC220& 35mA & 5V & 175mW\\

    \bottomrule
     & & &2833.4mW \\
    \bottomrule
    \end{tabular}%
    \caption{Energieverbrauch}
  \label{tab:budgetausgaben}%
\end{table}%


% Programmiersprache, Entwicklungsumgebung, abschätzung Datenmenge, Programmablauf (PAP), Datenverarbeitung (Steffen)
\subsection{Softwaredesign}
\subsubsection{Python als Programmiersprache}
Als Programmiersprache für das BeagleBone Black haben wir uns für Python entschieden. Es wäre ebenfalls möglich gewesen, den Mikrocontroller mit den Sprachen JavaScript, Java, C, C++, C\# und vielen weiteren Sprachen zu programmieren. Da es sich bei dem BeagleBone um ein Linux-basiertes Embedded System handelt, unterstützt es praktisch alle Programmiersprachen, sofern entsprechende Bibliotheken existieren. Allerdings haben wir uns aufgrund der Tatsache, dass Python im Gegensatz zu Java nicht objektorientiert geschrieben werden muss, für Python entschieden. Wir möchten auf der Hardwareseite möglichst auf objektorientierte Programmierung verzichten. Ein weiteres wichtiges Argument ist die gute Python-Bibliothek, welche von einer großen Community permanent gewartet und aktualisiert wird.
\subsubsection{Datenverarbeitung auf dem BeagleBone}
Die Datenverarbeitung auf dem BeagleBone verläuft relativ simpel. Zunächst werden alle von den Sensoren aufgezeichneten Daten, bei Sensoren mit I2C-Anbindung, mithilfe von Libraries, und bei den anderen mithilfe von Umrechnungsalgorithmen, gesammelt. Anschließend werden alle gesammelten Daten in einen JSON-String geparsed, welcher mithilfe von unserem Transceiver an die Bodenstation übermittelt wird. Diese übernimmt hieraufhin die weitere Verarbeitung und Darstellung der Messdaten. Zusätzlich werden die Daten auf dem internen Speicher des BeagleBones gespeichert.


% Fallschirm (Alexander F.)
\subsection{Bergungssystem}
Für unseren Landesystem haben wir uns entschieden unseren eigenen Fallschirm zu bauen. Die Hauptaufgabe ist es, eine weiche Landung auf dem Boden zu garantieren. Die Vorgabe war, dass der Fallschirm und die Dose eine Fallgeschwindigkeit von 15 Meter/Sekunde haben sollen. Unsere Testfallschirme, die wir auch schon im Vorjahr genutzt haben, waren für eine deutlich geringer Fallgeschwindigkeit ausgelegt. Dieses Jahr wollen wir eine neue Fallschirmart testen. Die Idee dabei ist, dass die acht Schnüre, welche den Fallschirm an der Dose befestigt, mit sehr luftdurchlässigen Stoff, sogenannter Gaze, ersetzt werden. Wir erhoffen uns dadurch eine stabilere Lage in der Luft und ein fortschrittlicheres Designe.

\subsubsection{Berechnungen}
Für den Bau des Fallschirm wissen wir bereits, das dieser mit v = 15m/s fallen soll. Außerdem haben wir einen bereits berechneten Strömungswiderstandskoeffizient (Cw) von 1,33. Für die Berechnung des Fallschirms wurde folgende Formel verwendet:
\[
Fw = Cw*\frac{1}{2}*roh*v²*A
\]
Fw ist die Strömungswiderstandskraft. Diese kann ermittelt werden, indem man die Fallwiederstandskarft einsetzt.
\[
Fw = m * g = 250g * 9,81\frac{m}{s²} = 2,4525\frac{m}{s²}*Kg
\]
Um die Größe des Fallschirms zu berechnen, kann man nun durch Einsetzten in die erste Formel diese nach A umstellen.
\[
A=\frac{2*2,4525\frac{m}{s²}*Kg}{Cw*Roh*v²}
\]
\[
A=\frac{2*2,4525\frac{m}{s²}*Kg}{1,33*1,2\frac{Kg}{m³}*15²\frac{m²}{s²}} = 0,01365m² \text{ oder } 136,5cm²
\]
Eine Fläche von 136,5 cm² entspricht einem Durchmesser von ca. 34 cm.

\subsubsection{Bau}
Beim Bau der Fallschirme haben wir den Stoff von Regenschirmen verwendet. Dieser Stoff ist bereits in einer Art Halbkugelform mit acht aneinandergefügten Panels. Er hat genug Widerstand um den Belastung des Fluges stand zu halten und ist sogar regendicht. Zuerst muss das Gestell vorsichtig vom Stoff herunter geschnitten werden. Danach muss in der Mitte des Stoff, wo alle acht Kanten aufeinander Treffen, ein rund 5 cm großes Loch geschnitten werden. Dort kann die gestauchte Luft leicht entweichen, statt am Rand unkontrolliert auszutreten. Würde sie dies tun, so könnte es leicht zu gefährlichen Turbulenzen kommen. Nun kann der Fallschirm in die entsprechende Größe zugeschnitten werden. Am Rand des Fallschirms sollte ein zusätzlicher ca. 1 cm Rand gelassen werden, der in den Fallschirm umgeklappt wird. Dieser muss mit einer Nähmaschine, auf dem Fallschirm, fest genäht werden. Dies dient dazu, die Struktur des Randes besser zu schützen. Am Rand befindet sich die höchste Wahrscheinlichkeit das durch eine kleine Kerbe ein kompletter Riss entstehen könnte. Nun kann auf diese Grundlage die Gaze genäht werden. Beachtet werden sollte zusätzlich, das beim Übergang von Fallschirm und Dose am Befestigungspunkt eine hohe Belastung auftritt. An diesem Punkt hat die Gaze einen gewaltigen Nachteil gegenüber der Schnüre. Hier kann es durch eine einmalige starke Belastung zum Riss kommen, der sich im Laufe des Fluges weiter ausbreiten kann. Daher wurde dort eine Verstärkung mit einem sehr rissfestem Material eingenäht, der die Spannung erst kompensiert und diese dann gut auf die Gaze verteilt.


% Import der Bodenstation Doku
\section{Beschreibung der Bodenstation}
\subsection{Einleitung}
In diesem Teil der Dokumentation werden wir die Bodenstation vorstellen, welche als Datenempfänger und als Datenverarbeitungsplattform fungiert. \\
Die Bodenstation wurde von Robin Bley, Marc Huisinga und Kevin Neumeyer entwickelt.

Die zentrale Aufgabe der Bodenstation ist es, die Daten, welche vom Satelliten gesammelt werden, zusätzlich sicher am Boden zu speichern, sollte der Satellit und damit auch die lokal gespeicherten Daten verloren gehen. \\
Zusätzlich zur Datensicherung erfüllt die Bodenstation die Aufgabe, die empfangenen Daten auf verschiedene Arten zu visualisieren und somit dem Nutzer direkt während der Datenübertragung die Möglichkeit zu verschaffen, die Daten zu beobachten und diese zu analysieren. \\
Die Bodenstation ermöglicht es außerdem, dass gesicherte Daten auch nach der Datenübertragung noch betrachtet und analysiert werden können. \\
Unser Ziel bei der Entwicklung der Bodenstation war es, eine modulare und anpassbare Plattform zu entwickeln, welche nicht nur mit unserem Satelliten, sondern mit vielen verschiedenen Satelliten genutzt werden kann, ohne dass ein großer Konfigurationsaufwand besteht. \\
Um dies zu ermöglichen, haben wir die Bodenstation in mehrere Dimensionen skalierbar entwickelt, was es im Endeffekt sehr einfach macht, neue Satelliten und verschiedene Übertragungsprotokolle zur Bodenstation hinzuzufügen.

\subsection{Verwendete Komponenten}
Zum Erreichen unserer Ziele haben wir verschiedene Komponenten verwendet, welche einerseits der Datenvisualisierung und -analyse dienen, andererseits aber auch der Entkopplung und skalierbaren für die Entwicklung dienen.

Für die Bodenstation haben wir folgende Komponenten verwendet:
\begin{description}
	\item[\href{http://www.oracle.com/technetwork/java/javase/downloads/jdk8-downloads-2133151.html}{Java}] ist eine objektorientierte Programmiersprache. Diese wurde verwendet, da jedes unserer Gruppenmitglieder damit vertraut ist. Die Version Java 8 wurde verwendet um mächtige funktionale Features zu nutzen
	\item[\href{https://netbeans.org/features/platform/}{Netbeans Platform}] das die Möglichkeit bietet, einfach eine integrierte, modulare und entkoppelte GUI-Applikation auf Basis von Java Swing zu entwickeln
	\item[\href{http://junit.org/}{JUnit}] ist ein Framework, welches zum Erstellen von automatisierten Softwaretests dient.
	\item[\href{http://fazecast.github.io/jSerialComm/}{JSerialComm (zum Start des Projektes noch serial-comm)}] ist eine Bibliothek, welche das Auslesen serieller Schnittstellen ermöglicht 
	\item[\href{http://worldwind.arc.nasa.gov/java/}{NASA World Wind}] ist eine Software, welche Satelliten- und Luftbilder auf einem virtuellen Erdball darstellt. Daten der Bodenstation werden mittels dieser Software in Relation zur Höhe in Echtzeit visualisiert.
	\item[\href{http://jchart2d.sourceforge.net/}{JChart2D}] ist eine Grafik-Bibliothek, welche zur grafischen Visualisierung von Daten dient. Mithilfe dieser Bibliothek werden zweidimensionale Graphen erzeugt, welche empfange Daten des Satelliten, in Relation zur Zeit oder anderen Daten, in einem Graphen darstellt
	\item[\href{http://www.json.org/}{JSON (JavaScript Object Notation)}] ist ein Datenformat, welches zum austausch von Daten zwischen Anwendungen angewand wird. JSON ermöglicht es Daten in verschiedenen Format in Textform zu speichern und sie wieder zurück in ihre uhrsprüngliche Form zu interpretieren. Dieses Datenformat wird in der Bodenstationsoftware genutzt Daten mit dem Sateliten auszutauschen, zu loggen, zu exportieren und zu importieren.
\end{description}

\subsection{Funktionen}
\subsubsection{Nutzerfreundlichkeit}

\subsubsection{Erweiterbarkeit}
Bei der Entwicklung der Bodenstation haben wir darauf geachtet, dass die Bodenstation auf einer skalierbaren Architektur aufgebaut ist. Dies ermöglicht es, leicht neue Module und Funktionalitäten zur Bodenstation hinzuzufügen, ohne dabei besonders viel Code abzuändern. Auf die folgenden Weisen ist die Bodenstation skalierbar:
\begin{itemize}
	\item Neue GUI-Komponenten können sehr leicht hinzugefügt werden. Das Erstellen und Einbinden eines GUI-Komponenten umfasst lediglich die Erstellung eines neuen Netbeans-TopComponents.
	\item Unterschiedliche Satelliten können ohne eine erneute Kompilierung hinzugefügt und verändert werden, indem man die Konfigurationsdateien der Bodenstation anpasst, welche zur Laufzeit von der Bodenstation geladen werden.
	\item Neue Konfigurationsformate können leicht hinzugefügt werden, indem man die neue Konfigurationsimplementation unter einem Interface in der Applikation hinzufügt.
	\item Es ist ohne viel Aufwand möglich, die verschiedenen Datenquellen, aus denen Daten bezogen werden, auszutauschen. Möchte man also die Daten von einer anderen Quelle als einem USB-Port beziehen, so ist dies leicht zu implementieren, indem man ledigliche eine neue DataSource implementiert und zur Applikation hinzufügt.
	\item Verschiedene Datenübertragungsformate können ebenfalls ohne viel Aufwand hinzugefügt werden, indem neue Formate unter dem DataFormat-Interface implementiert werden. Die Applikation ist also nicht auf JSON als Übertragungsformat limitiert.
	\item Die verschiedenen Datenempfänger können auch leicht angepasst werden, indem man neue Datenempfänger unter dem Receiver-Interface implementiert, wodurch die verschiedenen Logging-Formate erweitert werden können.
	\item Daten können aus beliebigen Dateiformaten importiert werden, was ebenfalls leicht erweiterbar ist, indem man neue Import-Dateiformate über das Importer-Interface implementiert.
	\item Export-Formate können leicht erweitert werden, indem man neue Export-Formate unter dem Exporter-Interface implementiert.
\end{itemize}

\subsubsection{Features}
Da die Software der Bodenstation auf dem Framework Netbeans Plantform basiert, lassen sich einzelne graphische Module kombinieren, welche sich per drag and drop verschieben lassen. Die Größe und Position dieser Module und des gesamten Frames lassen sich beliebig verändern. Des weiteren bietet die Software die Möglichkeit Daten von verschiedenen Satelliten zu empfangen. Empfangene Daten lassen sich mittels der graphischen Oberfläche in Graphen anzeigen, welche sich verschieden kombinieren lassen. Außerdem lassen sich die empfangenen Daten zwischenspeichern und anschließend in verschiedene Dateiformate Exportieren oder live in einer Datei loggen. Diese exportierten oder geloggten Daten lassen sich anschließend wieder einlesen und Anzeigen. CSV, Txt, JSON, KML und png sind Dateiformate, welche exportierbar sind. Davon lassen sich exportierte CSV und JSON Dateien wieder einlesen und Visualisieren. Txt Dateien werden Formatiert und somit gut leserlich für den Leser exportiert, während exportierte KML-Dateien per Google Earth geöffnet und graphisch Visualisiert werden können. Ein weiteres Feature der Bodenstationsoftware ist die Datenvisualisierung per Nasa World Wind als Modul in der graphischen Oberfläche. Diese Visualisierung zeigt den Flug des Satelliten auf einem virtuellen Globus, mittels Satelliten- und Luftbilder, und zeigt auf jeder gemessenen GPS Koordinate die gemessenen Werte der Sensoren des Satelliten. Unter anderem bietet dieses Modul der Software die Möglichkeit an der Virtuellen Erdkugel heran zu zoomen und einzelne Elemente dreidimensional darzustellen. Darstellungen mittels dieses virtuellen Erdballs sind sowohl in Echtzeit mittels eines Streams vom Satelliten als auch als import aus einer Datei möglich.


\subsection{Architektur}
Insgesamt ist die Architektur der Applikation um das GUI-Framework Netbeans Platform aufgebaut, da es die Nutzung von bestimmten Architekturen einfacher macht, mit Netbeans Platform zu arbeiten und alle Features von Netbeans Platform zu nutzen.

Insgesamt ist die Applikation in Module und Pakete aufgeteilt. Während normale Java-Projekte normalerweise lediglich in Pakete aufgeteilt sind, werden in Netbeans Platform die einzelnen Komponenten normalerweise in Module aufgeteilt. Jedes Modul verhält sich hierbei wie ein einzelnes Projekt, welches dann vom Hauptprojekt eingebunden wird. Dies fördert generell die Wiedernutzbarkeit der einzelnen Module, da sich Entwickler darüber Gedanken machen müssen, wie die einzelnen Module in verschiedenen Umgebungen verwendet werden können. \\
Den Aufbau der Netbeans Platform Modularchitektur ist auch im Anhang unter Abbildung \ref{nbp_modularchitektur} zu finden.

Anfänglich haben wir jeden einzelnen Teilkomponenten in ein Modul ausgelagert, was jedoch zu einer Menge Merge-Konflikten geführt hat, da Netbeans Platform für jedes einzelne Modul eigene Konfigurationsdateien generiert, über welche man nur schwer den Überblick behalten kann. Diese anfänglich Architektur wurde schlussendlich in eine Architektur mit nur drei Modulen umgewandelt: API, Core und GUI. \\
Das Core-Modul enthält die Programmlogik, welche sich hauptsächlich mit der Verarbeitung der Daten innerhalb der Input-Pipeline beschäftigt. \\
Das GUI-Modul enthält die verschiedenen GUI-Komponenten, welche sowohl Teil der allgemeinen Benutzeroberfläche sind, als auch Visualisierungskomponenten darstellen. \\
Innerhalb des API-Moduls befinden sich Interfaces und Utility-Klassen, welche sowohl vom Core-Modul als auch vom GUI-Modul genutzt werden. Das Core-Modul implementiert hierbei die Interfaces aus dem API-Modul, während das GUI-Modul die Programmlogik im Core-Modul lediglich entkoppelt über die Interfaces des API-Moduls anspricht. \\
Diese Architektur zeigt bereits, dass das GUI-Modul vom Core-Modul entkoppelt ist. Diese Entkopplung trägt stark zu der Erweiterbarkeit der Applikation bei. Die Architektur ähnelt hierbei der standardmäßigen ``Model, View, Controller''-Architektur, jedoch scheint innerhalb der Architektur kein Controller vorhanden zu sein. Auf den ersten Blick gesehen scheint es so, als spreche das GUI-Modul das Core-Modul direkt über das API-Modul an, jedoch sorgt Netbeans-Platform dafür, dass dem nicht so ist. Die von Netbeans Platform bereitgestellten Lookups, welche eine weitere Ebene der Entkoppelung darstellen, erfüllen in der Applikation die Aufgabe des Controllers. Durch die Lookups wird innerhalb des GUI-Moduls eine passende Klasse innerhalb des Core-Moduls über das Interface der Klasse im API-Modul geladen. Dank den Lookups und den Interfaces im API-Modul besteht absolut keinerlei Kopplung zwischen den einzelnen Klassen und Modulen: Die GUI kennt das Model nicht und das Model kennt die GUI nicht. \\
Die Modularchitektur der Bodenstation ist ebenfalls im Anhang unter Abbildung \ref{station_modularchitektur} zu finden.

% TODO: Input-Pipeline-Architektur

\include{./3_Beschreibung_der_Bodenstation/bodenstation_doku_softwaretestes}

\subsubsection{Nutzeranleitung}
\paragraph{Datenempfang}
Um Daten in Echtzeit zu empfangen muss eine Verbindung zu einem Satelliten aufgebaut werden. Um diese Verbindung aufzubauen wählt man unter dem Menüpunkt File den Unterpunkt Satellites aus. Dort ist es möglich unter add einen Satelliten hinzuzufügen. Wenn nun Daten empfangen werden sollen wählt man im selben Unterpunkt manage aus. Dort wählt man den Satelliten aus, von welchem man Daten empfangen will. Anschließend startet der Datenempfang.
\paragraph{Datenimport}
Um Daten aus einer Datei zu importieren wird Im Menüpunkt File der Untermenüpunkt Import Ausgewählt. Anschließend wird eine Datei ausgewählt, welche importiert werden soll. Mit der Bestätigung werden die Daten dieser Datei eingelesen.
\paragraph{Datenexport}
Für das Exportieren der Daten gilt, dass alle aktuell geladenen Daten exportiert werden. Darunter fallen entweder zwischengespeicherte Daten einer Liveübertragung oder Daten, welche aus einer Datei importiert werden.
Zum Exportieren der Daten wird im Menüpunkt File der Untermenüpunkt Export ausgewählt. Unter diesem Menüpunkt ist das Datenformat wählbar, in welches die gesammelten Daten gespeichert werden. Anschließend ist ein Pfad und eine Name wählbar unter dem die Datei gespeichert wird. Mit der Bestätigung werden die Daten exportiert.
\paragraph{Datenweiterleitung}
Per klick des auf das Icon des Servers in der Toolbar wird ein Server gestartet, empfangene Daten in Echtzeit an alle Clients versendet.
\paragraph{Oberflächenpersonalisierung}
Der Oberfläche können einzelne Komponenten hinzugefügt und entfernt werden. Diese Komponenten können unterschiedlich angeordnet werden. Um Komponenten wie z.B einen Graph hinzuzufügen wird entweder eine Graphvisualisierung oder eine Kartenvisualisierung hinzugefügt. Um einen der bestehenden Komponenten zu entfernen wird das Kreuz angeklickt, welches sich am Tab des Komponenten befindet. Per "drag and drop" können diese Komponenten neu angeordnet werden, dazu muss der Tab des Komponenten ausgewählt werden. In verschiedenen Bereichen können Komponenten angeheftet oder verschoben werden. Außerdem können diese übereinander verlagert werden um diese, in verschiedenen Tabs und in dem selben Bereich, zu verwalten.

\paragraph{Kartenvisualisierung}
Die Kartenvisualisierung startet über den Untermenüpunkt Map Vizualization im Menüpunkt Window. Geladene Werte werden dort angezeigt. Einzellen Messpunkt sind mit einem Punkt gekennzeichnet und mit einer Linie verbunden. Die mit Punkten gekennzeichneten Messwerte sind mit der linken Maustaste anklickbar. Mit einem Klick öffnen sich Details zu den angeklickt Messwerten.

\paragraph{Graphvisualisierung}
Um einen Graph zu erzeugen wählt man unter dem Menüpunkt Window Vizualization aus. Anschließend wird in der Oberfläche ein Graph erzeugt. Die Achsen des Graphs sind mit Sensorwerten belegbar. Um Belegung der Achsen zu verändern werden wählt man an den Achsen den jeweiligen Sensor aus und drückt den Button Ansicht aktualisieren.


\paragraph{Fenster zurücksetzen}
Die Anordnung der Komponenten der Oberfläche können im Menüpunkt window unter Reset Windows zurückgesetzt werden.

\paragraph{Beenden des Programms}
Um das Programm zu beenden gibt es zwei Möglichkeiten. Zum einen wird das Programm beendet, wenn das Kreuz am oberen rechten Rand der Oberfläche angeklickt wird. Zum andern kann das Programm über den Menüpunkt File geschlossen werden, in dem man dort Exit auswählt.

\include{./3_Beschreibung_der_Bodenstation/bodenstation_doku_kosten-/nutzenanalyse}



% Import der Projektplanung
\section{Projektplanung}
\subsection{Zeitplan der CanSat Vorbereitung}
Die Zeitplanung ist ausgerichtet für den Zeitpunkt der Abgabe unseres P5, da wir uns gewünscht haben, zu diesem Zeitpunkt mit dem Projekt fertig zu sein. Dieser Zeitplan wurde jedoch von Anfang an sehr kritisch gesehen. Daher ist es nicht verwunderlich, dass der Fortschritt des Projektes geringer ist, als er zum jetzigen Zeitpunkt eigentlich seien sollte. Dies ist jedoch nicht dramatisch, da bis zum Wettbewerb genügend Zeit ist die restlichen Arbeitspakete abzuarbeiten. Das gesamte Management der Arbeitspakete und des Zeitaufwandes wurde mit der Projektmanagementsoftware \href {www.redmine.org} {Redmine} erledigt. Da diese auf unserem Server unter \href{http://redmine.gamma-team.de}{redmine.gamma-team.de} erreichbar ist kann jedes Teammitglied zu jedem Zeitpunkt den Fortschritt der Arbeit verfolgen. Die Planung der beiden Halbgruppen ist größtenteils voneinander getrennt. Es gibt jedoch gemeinsame Meilensteine, welche von beiden Gruppen eingehalten werden sollen. Bevor die Arbeit der Halbgruppen begonnen hat gab es eine allgemeine Projektfindungsphase. In dieser Phase wurde ein grober Zeitplan festgelegt und es wurden alle relevanten Systeme (Webserver, Projektmanagementsoftware, GitLab etc.) aufgesetzt und eingerichtet um später einen reibungslosen Ablauf der Arbeitsphase zu garantieren. Die Idee und die Spezialisierung der Idee für das gesamte Projekt entstand ebenfalls in dieser Zeit. Anschließen wurde eine separate Zeitplanung in den beiden Halbgruppen erstellt, welche im Nachfolgenden erläutert wird.
\subsubsection{Zeitplan der Hardware Gruppe}
Innerhalb der Hardwaregruppe wurden versucht die meisten Aufgaben zu parallelisieren. Jedes Teammitglied hat sein eigenes spezielles Aufgabengebiet. jedoch herrscht trotzdem ein stetiger Austausch zwischen den Teammitgliedern. Grund für die Parallelisierung war, dass in unseren Augen die meisten Aufgaben  nur die Aufmerksamkeit einer Person benötigen. Es ist nur selten erforderlich, dass mehrere Teammitglieder an ein und dem selben Arbeitspaket arbeiten. Der gesamte Arbeitsprozess wurde in diverse Abschnitte gegliedert. Diese Abschnitte lassen sich auch im GANTT Diagramm im Anhang dieses Dokumentes wiederfinden. Bei den Abschnitten handelt es sich um folgende:
\begin{itemize}
\item Planung: Erstellung von Arbeitspaketen, sowie eine Verteilung dieser und eine Erstellung diverser Diagramme
\item Fallschirm: Gestaltung und Bau des Bergungssystems.
\item Sensorik: Dieser Abschnitt behandelt das Heraussuchen, Bestellen und Testen passender Sensoren für unser Projekt.
\item Beagleboard: Festlegung der Programmiersprache, IDE und der Recherche zu den elektrotechnischen Eigenschaften des Boards
\item Dose: Design und Bau der Hülle und der Deckel der Dose
\item Dosenmanagement: Design und Bau des inneren der Dose, sowie die Integration der Sensoren in das Gesamtsystem
\end{itemize}

Die einzelnen Abschnitte sind in diverse Arbeitspakete unterteilt, Personen zugewiesen und mit einem Zeitraum versehen.
\subsection{Einschätzung der Mittel}
\subsubsection{Budget}
% \begin{tabular}{p{1,5cm}p{1,5cm}p{3,5cm}p{6,5cm}rrrl}
\label{subsubsec:Budget}

Um das CanSat Projekt zu finanzieren konnten wir aktuell noch keine Sponsoren finden. Jedoch konnten wir uns mit unserem Schulverein verständigen, welcher uns finanziell unterstützen wird. Da wir nicht auf das T-Minus Kitt zurückgreifen sondern stattdessen ein anderes Mikrocontroller Board verwenden können wir ungefähr 150  \euro  sparen. Der 200 \euro Watterot Gutschein, welcher vom Wettbewerb gestellt wird, ist in unseren Rechnung noch nicht inbegriffen. Dies liegt daran, dass noch nichts bei Watterot bestellt wurde, bzw. die Bestellung lange vor der Annahme am Wettbewerb getätigt wurde.
Im Nachfolgenden sind alle Ausgaben und Einnahmen aufgelistet.
\begin{table}[htbp]
  \centering
  \caption{Ausgaben}
    \begin{tabular}{p{1,7cm}p{1,5cm}p{3,5cm}p{6,5cm}rrrl}
    \toprule
    \textbf{Ausgabe} & \textbf{Datum} & \textbf{Empfänger} & \textbf{Grund} \\
    \midrule
    -12,16 \euro  & 08.01.2015 & Watterott & BMP180 Breakout \\
    -28,99 \euro  & 09.01.2015 & eBay - rcskymodel & Ultimate GPS \\
    -14,32 \euro  & 10.01.2015 & Spark Fun Electronics & UV-Sensor \\
    -51,99 \euro  & 10.01.2015 & Amazon & Beagle Bone Black \\
    -17,30 \euro  & 01.12.2014 & eBay - hdt-preiswert & 
GFK-Set 1kg Polyesterharz + 20g Härter + $2m^2$ Glasfasermatte \\
    -3,54 \euro  & 23.03.2015 & toom baumarkt & 6 x Schleifpapier \\
    -3,79 \euro  & 23.03.2015 & toom baumarkt & Filzrolle \\
    -4,49 \euro  & 23.03.2015 & toom baumarkt & Plüschwalzen \\
    -2,19 \euro  & 23.03.2015 & toom baumarkt & Mundschutz \\
    -1,99 \euro  & 23.03.2015 & toom baumarkt & Farbwanne \\
    -4,99 \euro  & 23.03.2015 & toom baumarkt & Einmalhandschuhe \\
    \bottomrule
    - 145,75 \euro & & & \\
    \bottomrule
    \end{tabular}%
  \label{tab:budgetausgaben}%
\end{table}%

\begin{table}[htbp]
  \centering
  \caption{Einnahmen}
    \begin{tabular}{p{1,7cm}p{1,5cm}p{3,5cm}p{6,5cm}rrrl}
    \toprule
    \multicolumn{1}{c}{\textbf{Einnahmen}} & \textbf{Datum} & \textbf{Absender} & \textbf{Grund} \\
    \midrule
              17,30 \euro  & 01.12.2014 & Alexander Brennecke & GFK-Kauf \\
           107,46 \euro  & 10.01.2015 & Alexander Brennecke & Sensorenkauf \\
              20,99 \euro  & 23.03.2015 & Alexander Brennecke & toom Einkauf \\
    \bottomrule
    145,75 \euro & & & \\
    \bottomrule
    \end{tabular}%
  \label{tab:budgeteinnahmen}%
\end{table}%

\subsubsection{Externe Unterstützung}
Externe Unterstützung erhielten wir von vielen Lehrern unserer Schule, welche uns Fragen zur Elektrotechnik und Softwareprogrmmierung beantworten konnten. Zusätzlich haben wir finanzielle Unterstützung durch den Schulverein unserer Schule erhalten (siehe~\ref{subsubsec:Budget}).
Unterstützung außerhalb unserer Schule erhileten wir durch folgende Personen/Organisationen:

\begin{itemize}
	\item Das \href{https://www.hackerspace-bremen.de/}{Hackerspace Bremen e.V.}, welches uns ihren 3D-Drucker zur Verfügung gestellt hat. Zusätzlich konnten wir dort unsere Platine ätzen.
	\item \href{http://de.wikipedia.org/wiki/Martin_Schneider_(Nachrichtentechniker)} {Prof. Martin Schneider} von von dem Hochfrequenzlabor der Universität Bremen, welcher uns geholfen hat unsere Antenne an die Frequenz und die Wellenimpedanz anzupassen.
	\item  Das Umweltlabor der \href{http://www.atlas-elektronik.com/atlas-elektronik/}{Atlas Elektronik GmbH} hat uns geholfen den CanSat, hinsichtlich seiner Stabilität, zu testen und die Sensoren korrekt zu kalibrieren.
\end{itemize}

\subsubsection{Testkonzept}

% Import der Oeffentlichkeitsarbeit
\section{Öffentlichkeitsarbeit}
\subsection{Website}
Unsere Website \href{www.team-gamma.de}{Team Gamma} wurde bereits für den europäischen CanSat Wettbewerb 2014 verwendet. Diese haben wir weiter geführt und dort in unregelmäßigen Abständen aktuelle Informationen über das Projekt veröffentlicht. Da die Website durch den europäischen Wettbewerb bei anderen europäischen Teams bekannt ist wird die Website in Englisch geführt. Man findet dort zusätzlich einige Dokumente, Fotos und Videos. Die Informationen auf der Website sind meist relativ detailliert verfasst. 

\subsection{Schülerzeitung}
In der Schülerzeitung unserer Schule sind bereits diverse Artikel über unser Projekt erschienen und sollen auch in Zukunft erscheinen. Diese Artikel handeln zumeist von dem Wettbewerb selber und gehen weniger auf die technischen Details ein.

\subsection{Präsentationen}
Da wir das CanSat Projekt bereits seit einiger Zeit betreiben präsentierten wir es mehrere Male vor unserer Klasse. Dies kommt zum Beispiel dann vor, wenn wir Teile des Projektes in Schulprojekte einfließen lassen. Zusätzlich haben wir, beispielsweise am Tag der offenen Tür unserer Schule, diversen Schulbesuchern das Projekt und den Wettbewerb näher gebracht.

\subsection{Ausstellung am MINT-Projekttag unserer Schule}
Im Schuljahr 2015/2016 findet an unserer Schule ein Tag der MINT Projekte statt. Dieser Tag wird von einer Schülergruppe unserer Parallelklasse organisiert. Wir möchten an diesem Tag natürlich auch unser Projekt vorstellen.

\subsection{Logo}
Das Logo wurde ebenfalls aus Gründen der Wiedererkennbarkeit aus dem vorherigen Jahr übernommen. Das Aussehen des Logos wurde von drei Faktoren beeinflusst:
\begin{itemize}
	\item Das Zeichen in der Mitte soll dem Gamma Logo ähneln, welches zu unserem Teamnamen passt
	\item Das Zeichen soll zusätzlich, wenn man es um $180^\circ$ dreht, dem Lambda Logo ähneln. Da wir bei dem Entwurf unserer Antenne immer wieder auf Lambda gestoßen sind, sind daraus diverse interne Späße entstanden, die wir in das Logo einfließen lassen wollen.
	\item Das Logo des Computerspiel Halflife, welches von einigen Teammitgliedern gespielt wird
\end{itemize}

% Import der Anforderungen
\section{Anforderungen}
\begin{table}[htbp]
  \centering
  
    \begin{tabular}{l|r}
    \toprule
    \textbf{Anforderung} & \textbf{Messwert}  \\
    \midrule
    Masse des CanSat  & xy mm \\
    Höhe des CanSat	  & xy mm\\
	Länge des Bergungssystems (vgl. Pkt 2 Anhang 1)  & xy mm \\
	Planmäßige Flugzeit  & xy Sekunden \\
    Berechnete Sinkgeschwindigkeit  & xy m/S \\
    Genutzte Funkfrequenz & xy mHz \\
    Energieverbrauch & xy mW \\
    Gesamtkosten & xy \euro \\
    \bottomrule
    \end{tabular}%
    \caption{Anforderungen an den CanSat}
  \label{tab:anforderungen}%
\end{table}%

% Import der Reflexion
\section{Reflexion des Projektverlaufes}
\subsection {Reflexion der Hardwaregruppe}
Als wir angefangen haben das gesamte Projekt zu planen haben wir uns als Ziel gesetzt Ende Mai fertig zu sein. Dieses Datum haben wir Aufgrund der Abgabe unseres P5 gewählt, für welches wir das CanSat Projekt ebenfalls einreichen wollen. Uns war bewusst, dass dies ein sehr hoch gestecktes Ziel ist. Im Nachhinein haben wir relativ schnell gemerkt, dass wir dieses Ziel nicht erreichen können. Diese Verzögerung wurde durch mehrere Faktoren hervorgerufen. Dazu zählt der enorm hohe Anspruch den wir uns selber gesetzt haben. Dieser hatte immer wieder zur Folge, dass viele Dinge mehrfach oder gründlicher gemacht werden mussten, als es zu Anfang geplant war. Zum anderen haben wir verhältnismäßig lange gebraucht um uns auf eine finale Idee festzulegen und diese zu präzisieren. Da wir uns jedoch kontinuierlich zum arbeiten getroffen haben konnten wir dennoch gute Fortschritte erzielen. Wir lagen zwar die meiste Zeit über hinter unserem Zeitplan, konnten jedoch die Reihenfolge der zu bearbeitenden Aufgabenpakete größtenteils einhalten.
\subsection {Reflexion der Softwaregruppe}
\subsection {Reflexion der Zusammenarbeit zwischen den Teams}
Da der inhaltliche Schwerpunkt der beiden Teams relativ wenig miteinander zu tun hat sollte es theoretisch relativ wenige Berührungspunkte geben. Dies war bei unserer Projektarbeit jedoch nicht so. Da die Arbeit der beiden Halbgruppen zur gleichen Zeit in der gleichen Räumlichkeit stattfand war es oft so, dass teamübergreifend  diskutiert wurde. Dies hat den Vorteil, dass beide Teams nochmal einen anderen Blick auf eventuelle Problemstellungen bekommen und so einfache oder bessere Lösungen für Probleme finden können. Zusätzlich lief die Absprache über den Datenaustausch zwischen Bodenstation und CanSat sehr gut. Die beiden Teams haben also hervorragend kooperiert und gemeinsam versucht ein bestmögliches Gesamtprodukt zu erschaffen.

% Import des Anhanges
\section{Anhang}

\subsection{Einleitung}
\subsubsection{Blockdiagramm}
\begin{figure}[htbp]
	\centering
	\includegraphics[width=0.8\textwidth]{8_Anhang/Blockdiagramm.png}
	\caption{Blockdiagramm vom CanSat}
	\label{blockdiagramm}
\end{figure}

\newpage

\subsection{GANTT-Diagramme}
\subsubsection {Hardware-GANTT}
\begin{figure}[htbp]
	\centering
	\includegraphics[trim = 10mm 50mm 20mm 65mm, clip,width=0.8\textwidth]{8_Anhang/hardware-gantt-1.png}
	\label{gantt_hardware_1}
\end{figure}
\vspace{-2cm}
\begin{figure}[htbp]
	\centering
	\includegraphics[trim = 11mm 350mm 20mm 40mm, clip,width=0.8\textwidth]{8_Anhang/hardware-gantt-2.png}
	\caption{Das GANTT-Diagramm der Hardware-Gruppe}
	\label{gantt_hardware_2}
\end{figure}

\newpage
\subsubsection {Bodenstation-GANTT}
\begin{figure}[H]
	\centering
	\includegraphics[trim = 25mm 50mm 46mm 65mm, clip,width=0.8\textwidth]{8_Anhang/bodenstation-gantt-1.png}
	\label{gantt_hardware_1}
\end{figure}
\vspace{-1,3cm}
\begin{figure}[H]
	\centering
	\includegraphics[trim = 30mm 50mm 45mm 40mm, clip,width=0.8\textwidth]{8_Anhang/bodenstation-gantt-2.png}
	\label{gantt_hardware_2}
\end{figure}
\vspace{-1,1cm}
\begin{figure}[H]
	\centering
	\includegraphics[trim = 30mm 200mm 45mm 40mm, clip,width=0.8\textwidth]{8_Anhang/bodenstation-gantt-3.png}
	\caption{Das GANTT-Diagramm der Bodenstation}
	\label{gantt_hardware_3}
\end{figure}
\newpage
\subsubsection {Android-App-GANTT}
\begin{figure}[H]
	\centering
	\includegraphics[trim = 30mm 200mm 45mm 40mm, clip,width=0.8\textwidth]{8_Anhang/android-app-gantt.png}
	\caption{Das GANTT-Diagramm der Android-App}
	\label{gantt_hardware_3}
\end{figure}

\newpage
\vspace{-2cm}
\subsection{Der CanSat}

\begin{figure}[H]
	\centering
	\includegraphics[trim = 11mm 335mm 20mm 210mm, clip, width=0.8\textwidth]{8_Anhang/dose.jpg}
	\caption{Die Hülle und ein Dosendeckel}
	\label{pic_dose}
\end{figure}

\begin{figure}[H]
	\centering
	\includegraphics[ width=0.8\textwidth]{8_Anhang/Wand_3D.PNG}
	\caption{Screenshot der Zwischenwand aus Sketchup}
	\label{pic_wand_3d}
\end{figure}

\newpage

\begin{figure}[h] 
      \centering 
      \includemovie[ 
        poster,controls,
       3Djscript=2_Beschreibung_des_CANSAT/Encompass.js
      ]{0.8\textwidth}{15cm}{2_Beschreibung_des_CANSAT/CanSat_2015.u3d} 
      \caption{Der Satellit (Diese Zeichnung ist möglicherweiße nicht sichtbar, da es eine 3D-Zeichnung ist. Bitte verwenden Sie den \href{https://get.adobe.com/reader/?loc=de}{Adobe Acrobat Reader})}\label{pic_3d} 
\end{figure} 

\newpage

\begin{figure}[H]
	\centering
	\includegraphics[trim = 60mm 100mm 55mm 100mm, clip, width=0.8\textwidth]{8_Anhang/Schaltplanv1.png}
	\caption{Der Schaltplan der Sensorikplatine}
	\label{pic_schaltplan}
\end{figure}

\newpage

\begin{figure}[H]
	\centering
	\includegraphics[trim = 200mm 300mm 200mm 280mm, clip,width=0.8\textwidth]{8_Anhang/Layoutv1.png}
	\caption{Das Layout der Sensorikplatine}
	\label{pic_layout}
\end{figure}

\newpage

\begin{figure}[H]
	\centering
	\includegraphics[ width=0.8\textwidth]{8_Anhang/fallschirmSkizze.PNG}
	\caption{Skizze des Fallschirms}
	\label{pic_fallschrimskizze}
\end{figure}

\newpage

\subsection{Bodenstationsarchitektur}
\begin{figure}[H]
	\centering
	\includegraphics[width=0.8\textwidth]{3_Beschreibung_der_Bodenstation/NBP_Modularchitektur.png}
	\caption{Modularchitektur von Netbeans Platform}
	\label{nbp_modularchitektur}
\end{figure}

\begin{figure}[H]
	\centering
	\includegraphics[width=0.8\textwidth]{3_Beschreibung_der_Bodenstation/Bodenstation_Modularchitektur.png}
	\caption{Modularchitektur der Bodenstation}
	\label{station_modularchitektur}
\end{figure}
\vspace{-5cm}

\newpage

\begin{figure}[H]
	\centering
	\includegraphics[width=0.8\textwidth]{3_Beschreibung_der_Bodenstation/Input-Pipeline_Architektur.png}
	\caption{Architektur der Input-Pipeline}
	\label{inputpipeline}
\end{figure}

\newpage
\subsection{Protokolle}
Auf den nachfolgenden Seiten finden sich Protokolle diverser Meetings. Diese Protokolle wurden mal mit mehr, und mal mit weniger Mühe und Aufwand angefertigt. Uns war lediglich wichtig, dass es Protokolle gibt, an denen wir unsere Arbeit belegen können, und an denen bereits getroffene Entscheidungen nachvollzogen werden können.
\includepdf[pages=-]{8_Anhang/Protokolle/140625_Meeting_Protokoll}%

\newpage
\includepdf[pages=-]{8_Anhang/Protokolle/140921_Meeting_Protokoll}%

\newpage
\includepdf[pages=-]{8_Anhang/Protokolle/141022_Meeting_Protokoll}%

\newpage
\begin{figure}[H]
	\centering
	\includegraphics[width=0.8\textwidth]{8_Anhang/Protokolle/141203_Meeting_Protokoll.png}%
	\label{pic_protokoll_140203}
\end{figure}

\newpage
\includepdf{8_Anhang/Protokolle/150106_Meeting_Protokoll}%
\newpage

\includepdf{8_Anhang/Protokolle/150204_Meeting_Protokoll}%
\newpage
\includepdf{8_Anhang/Protokolle/150210_Meeting_Protokoll}%
\newpage
\includepdf{8_Anhang/Protokolle/150217_Meeting_Protokoll}%
\newpage
\includepdf{8_Anhang/Protokolle/150303_Meeting_Protokoll}%
\newpage
\includepdf{8_Anhang/Protokolle/150310_Meeting_Protokoll}%
\newpage
\includepdf{8_Anhang/Protokolle/150317_Meeting_Protokoll}%
\newpage
\includepdf{8_Anhang/Protokolle/150412_Meeting_Protokoll}%
\newpage
\includepdf{8_Anhang/Protokolle/150419_Meeting_Protokoll}%
\newpage
\includepdf{8_Anhang/Protokolle/150420_Meeting_Protokoll}%
\newpage
\includepdf{8_Anhang/Protokolle/150426_Meeting_Protokoll}%
\newpage
\includepdf{8_Anhang/Protokolle/150427_Meeting_Protokoll}%
\newpage
\includepdf{8_Anhang/Protokolle/150429_Meeting_Protokoll}%
\newpage
\includepdf{8_Anhang/Protokolle/150506_Meeting_Protokoll}%
\newpage
\includepdf{8_Anhang/Protokolle/150512_Meeting_Protokoll}%
\newpage
\includepdf{8_Anhang/Protokolle/150520_Meeting_Protokoll}%

\begin{comment}
0

{140921,141022,141202,141203,150106,150204,150210,150217,150303,150310,150317,150412,150419,150420,150426,150429,150506,150512,150520,150522} {

\foreach \i in {00, ..., 999999}%
    \edef\FileName{\i\_Meeting\_Protokoll}%     The % here are necessary to eliminate any
    \FileName%
    \IfFileExists{\FileName}{%  spurious spaces that may get inserted
    \includepdf[width=0.8\textwidth]{8_Anhang/Protokolle/\FileName}%
}%
}%

\end{comment}



\end{document}