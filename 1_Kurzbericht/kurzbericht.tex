\section{Kurzbericht}
In den vergangenen Monaten haben wir uns damit beschäftigt, unser Projekt fertigzustellen und durchzutesten. \\
Im Bereich der Hardware sind bereits der Fallschirm, die Einrichtung unseres Mikrocontrollers, die eigentliche Dose, die Sensorik, die eigentliche Sensorikplatine, das Innere der Dose und die Antenne fertiggestellt. Aktuell beschäftigen wir und mit der Integration der Komponenten. Einige Bereiche sind bereits durchgetestet, während andere Bereiche, welche das fertige Gesamtsystem erfordern, aktuell noch getestet werden. \\
Die Desktopapplikation ist mittlerweile komplett fertiggestellt und wird lediglich noch getestet und an einigen Stellen verbessert. Es ist möglich, die Daten, beispielsweise mithilfe einer dreidimensionalen Kartenvisualisierung oder eines Graphen, darzustellen. Die Benutzeroberfläche kann beliebig angepasst und konfiguriert werden. Eines der wichtigsten Features ist die Kompatibilität zu anderen CanSats. Über die Benutzeroberfläche ist es möglich, beliebige CanSats und Übertragungsformate zu integrieren. So kann unsere Desktopapplikation auch von anderen Teams genutzt werden. \\
Auch die Android-Applikation wurde komplett fertiggestellt. Sie ist dazu in der Lage, Daten über einen Hotspot von der Bodenstation zu beziehen und diese Daten in der Applikation mithilfe von Graphen darzustellen. Die Android-Applikation wurde bereits ausgiebig getestet. \\
Aktuell sind wir also dabei, die Hardware zu integrieren und das Gesamtsystem ausführlich zu testen, was darauf schließen lässt, dass das Projekt sehr bald als abgeschlossen erklärt werden kann.