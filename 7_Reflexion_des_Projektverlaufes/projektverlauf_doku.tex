\section{Reflexion des Projektverlaufes}
\subsection {Reflexion der Hardwaregruppe}
Als wir angefangen haben das gesamte Projekt zu planen haben wir uns als Ziel gesetzt Ende Mai fertig zu sein. Dieses Datum haben wir Aufgrund der Abgabe unseres P5 gewählt, für welches wir das CanSat Projekt ebenfalls einreichen wollen. Uns war bewusst, dass dies ein sehr hoch gestecktes Ziel ist. Im Nachhinein haben wir relativ schnell gemerkt, dass wir dieses Ziel nicht erreichen können. Diese Verzögerung wurde durch mehrere Faktoren hervorgerufen. Dazu zählt der enorm hohe Anspruch den wir uns selber gesetzt haben. Dieser hatte immer wieder zur Folge, dass viele Dinge mehrfach oder gründlicher gemacht werden mussten, als es zu Anfang geplant war. Zum anderen haben wir verhältnismäßig lange gebraucht um uns auf eine finale Idee festzulegen und diese zu präzisieren. Da wir uns jedoch kontinuierlich zum arbeiten getroffen haben konnten wir dennoch gute Fortschritte erzielen. Wir lagen zwar die meiste Zeit über hinter unserem Zeitplan, konnten jedoch die Reihenfolge der zu bearbeitenden Aufgabenpakete größtenteils einhalten.
\subsection {Reflexion der Softwaregruppe}
\subsection {Reflexion der Zusammenarbeit zwischen den Teams}
Da der inhaltliche Schwerpunkt der beiden Teams relativ wenig miteinander zu tun hat sollte es theoretisch relativ wenige Berührungspunkte geben. Dies war bei unserer Projektarbeit jedoch nicht so. Da die Arbeit der beiden Halbgruppen zur gleichen Zeit in der gleichen Räumlichkeit stattfand war es oft so, dass teamübergreifend  diskutiert wurde. Dies hat den Vorteil, dass beide Teams nochmal einen anderen Blick auf eventuelle Problemstellungen bekommen und so einfache oder bessere Lösungen für Probleme finden können. Zusätzlich lief die Absprache über den Datenaustausch zwischen Bodenstation und CanSat sehr gut. Die beiden Teams haben also hervorragend kooperiert und gemeinsam versucht ein bestmögliches Gesamtprodukt zu erschaffen.