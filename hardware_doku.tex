\section{Der CanSat}
\subsection{Einleitung}
Wir haben uns für den Satelliten überlegt, dass dieser so weit wie möglich individuell sein sollte. Daher greifen wir nicht auf das, vom Wettbewerb bereitgestelte T-Minus CanSat Kit zurück. Stattdessen haben wir uns im Detail überlegt, welche Sensoren unseren Erwartungen entsprechen und wie wir diese bestmöglich innerhalb der Dose plazieren können. Zusätzlich möchten wir nicht auf eine Cola-Dose als Hülle zurück greifen, sondern möchten auch hier unser eigenes Design erschaffen.

\subsection {Hülle und Platzmanagement}
Wir haben uns dazu entschieden, die äußere Hülle aus GFK (Glasfaser verstärkter Kunststoff) zu fertigen. Dieser hat die Eigenschaften, dass er bei einem sehr geringen Gewicht, und bei einer geringen Wandstärke trotzdem eine gewisse Stabilität aufweißt. Aus dem GFK haben wir eine Röhre mit einem Innendurchmesser von 31,5 mm und einem Außendurchmesser von 33,5 mm laminiert. Diese Röhre wurde auf eine Länge von 111 mm gekürzt und gefeilt.  